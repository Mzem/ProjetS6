%% Generated by Sphinx.
\def\sphinxdocclass{report}
\documentclass[letterpaper,10pt,french]{sphinxmanual}
\ifdefined\pdfpxdimen
   \let\sphinxpxdimen\pdfpxdimen\else\newdimen\sphinxpxdimen
\fi \sphinxpxdimen=.75bp\relax

\usepackage[utf8]{inputenc}
\ifdefined\DeclareUnicodeCharacter
 \ifdefined\DeclareUnicodeCharacterAsOptional\else
  \DeclareUnicodeCharacter{00A0}{\nobreakspace}
\fi\fi
\usepackage{cmap}
\usepackage[T1]{fontenc}
\usepackage{amsmath,amssymb,amstext}
\usepackage{babel}
\usepackage{times}
\usepackage[Sonny]{fncychap}
\usepackage[dontkeepoldnames]{sphinx}

\usepackage{geometry}

% Include hyperref last.
\usepackage{hyperref}
% Fix anchor placement for figures with captions.
\usepackage{hypcap}% it must be loaded after hyperref.
% Set up styles of URL: it should be placed after hyperref.
\urlstyle{same}

\addto\captionsfrench{\renewcommand{\figurename}{Fig.}}
\addto\captionsfrench{\renewcommand{\tablename}{Tableau}}
\addto\captionsfrench{\renewcommand{\literalblockname}{Code source}}

\addto\extrasfrench{\def\pageautorefname{page}}

\setcounter{tocdepth}{1}



\title{File Rouge Documentation}
\date{mai 24, 2017}
\release{1.0}
\author{PAILLEUX Jean-Didier, ZEMNI Malek, KLOTZ Sonny}
\newcommand{\sphinxlogo}{\vbox{}}
\renewcommand{\releasename}{Version}
\makeindex

\begin{document}

\maketitle
\sphinxtableofcontents
\phantomsection\label{\detokenize{index::doc}}


Contents:


\chapter{gestionFlux.py}
\label{\detokenize{gestionFlux:bienvenue-sur-la-documentation-du-projet-file-rouge}}\label{\detokenize{gestionFlux:module-interface_web.gestionFlux}}\label{\detokenize{gestionFlux:gestionflux-py}}\label{\detokenize{gestionFlux::doc}}\index{interface\_web.gestionFlux (module)}

\section{Le module \sphinxstyleliteralintitle{Gestion des flux}}
\label{\detokenize{gestionFlux:le-module-gestion-des-flux}}\index{fenetre\_choix\_fichier() (dans le module interface\_web.gestionFlux)}

\begin{fulllineitems}
\phantomsection\label{\detokenize{gestionFlux:interface_web.gestionFlux.fenetre_choix_fichier}}\pysiglinewithargsret{\sphinxcode{interface\_web.gestionFlux.}\sphinxbfcode{fenetre\_choix\_fichier}}{}{}
Fonction qui affiche le template « choix\_fichier.html » lorsque la requette HTTP « /fenetre\_choix\_fichier/ » est indiquée.
\begin{quote}\begin{description}
\item[{Retourne}] \leavevmode
retourne le template « choix\_fichier.html »

\end{description}\end{quote}

\end{fulllineitems}

\index{fenetre\_role\_choix\_colonne() (dans le module interface\_web.gestionFlux)}

\begin{fulllineitems}
\phantomsection\label{\detokenize{gestionFlux:interface_web.gestionFlux.fenetre_role_choix_colonne}}\pysiglinewithargsret{\sphinxcode{interface\_web.gestionFlux.}\sphinxbfcode{fenetre\_role\_choix\_colonne}}{\emph{file}}{}
Fonction qui affiche le template « role\_choix\_colonne.html » lorsque la requette HTTP « /fenetre\_role\_choix\_colonne/ » est indiquée.
\begin{quote}\begin{description}
\item[{Paramètres}] \leavevmode
\sphinxstyleliteralstrong{file} \textendash{} représente le nom du fichier chargé

\item[{Retourne}] \leavevmode
retourne le template « role\_choix\_colonne.html »

\end{description}\end{quote}

\end{fulllineitems}

\index{fenetre\_resultat\_ADD() (dans le module interface\_web.gestionFlux)}

\begin{fulllineitems}
\phantomsection\label{\detokenize{gestionFlux:interface_web.gestionFlux.fenetre_resultat_ADD}}\pysiglinewithargsret{\sphinxcode{interface\_web.gestionFlux.}\sphinxbfcode{fenetre\_resultat\_ADD}}{}{}
Fonction qui affiche le template « resultat\_ADD.html » lorsque la requette HTTP « /fenetre\_resultat\_ADD/ » est indiquée.
\begin{quote}\begin{description}
\item[{Retourne}] \leavevmode
retourne le template « resultat\_ADD.html »

\end{description}\end{quote}

\end{fulllineitems}

\index{remove() (dans le module interface\_web.gestionFlux)}

\begin{fulllineitems}
\phantomsection\label{\detokenize{gestionFlux:interface_web.gestionFlux.remove}}\pysiglinewithargsret{\sphinxcode{interface\_web.gestionFlux.}\sphinxbfcode{remove}}{\emph{file}}{}
Fonction qui supprime le fichier uploadé
\begin{quote}\begin{description}
\item[{Param}] \leavevmode
file de type str correspondant au nom du fichier csv

\item[{Retourne}] \leavevmode
redirige vers la route index

\end{description}\end{quote}

\end{fulllineitems}

\index{iStats() (dans le module interface\_web.gestionFlux)}

\begin{fulllineitems}
\phantomsection\label{\detokenize{gestionFlux:interface_web.gestionFlux.iStats}}\pysiglinewithargsret{\sphinxcode{interface\_web.gestionFlux.}\sphinxbfcode{iStats}}{}{}
\end{fulllineitems}

\index{timeSeries() (dans le module interface\_web.gestionFlux)}

\begin{fulllineitems}
\phantomsection\label{\detokenize{gestionFlux:interface_web.gestionFlux.timeSeries}}\pysiglinewithargsret{\sphinxcode{interface\_web.gestionFlux.}\sphinxbfcode{timeSeries}}{}{}
\end{fulllineitems}

\index{distribution() (dans le module interface\_web.gestionFlux)}

\begin{fulllineitems}
\phantomsection\label{\detokenize{gestionFlux:interface_web.gestionFlux.distribution}}\pysiglinewithargsret{\sphinxcode{interface\_web.gestionFlux.}\sphinxbfcode{distribution}}{}{}
\end{fulllineitems}

\index{distributionCumulative() (dans le module interface\_web.gestionFlux)}

\begin{fulllineitems}
\phantomsection\label{\detokenize{gestionFlux:interface_web.gestionFlux.distributionCumulative}}\pysiglinewithargsret{\sphinxcode{interface\_web.gestionFlux.}\sphinxbfcode{distributionCumulative}}{}{}
\end{fulllineitems}

\index{sauvegardeResultats() (dans le module interface\_web.gestionFlux)}

\begin{fulllineitems}
\phantomsection\label{\detokenize{gestionFlux:interface_web.gestionFlux.sauvegardeResultats}}\pysiglinewithargsret{\sphinxcode{interface\_web.gestionFlux.}\sphinxbfcode{sauvegardeResultats}}{}{}
\end{fulllineitems}



\chapter{choixFichier.py}
\label{\detokenize{choixFichier:module-interface_web.choixFichier}}\label{\detokenize{choixFichier:choixfichier-py}}\label{\detokenize{choixFichier::doc}}\index{interface\_web.choixFichier (module)}

\section{Le module \sphinxstyleliteralintitle{Fenêtre choix fichier}}
\label{\detokenize{choixFichier:le-module-fenetre-choix-fichier}}\index{FileWithSGF() (dans le module interface\_web.choixFichier)}

\begin{fulllineitems}
\phantomsection\label{\detokenize{choixFichier:interface_web.choixFichier.FileWithSGF}}\pysiglinewithargsret{\sphinxcode{interface\_web.choixFichier.}\sphinxbfcode{FileWithSGF}}{}{}
Fonction qui se charge de l’upload d’un fichier lors du parcours dans le système de gestions de fichiers.
\begin{quote}\begin{description}
\item[{Retourne}] \leavevmode
redirige la page web vers la Fenêtre choix fichier avec comme parmètre le chemin du fichier uploadé.

\end{description}\end{quote}

\end{fulllineitems}

\index{FileWithDragDrop() (dans le module interface\_web.choixFichier)}

\begin{fulllineitems}
\phantomsection\label{\detokenize{choixFichier:interface_web.choixFichier.FileWithDragDrop}}\pysiglinewithargsret{\sphinxcode{interface\_web.choixFichier.}\sphinxbfcode{FileWithDragDrop}}{}{}
Fonction qui se charge de l’upload d’un fichier après avoir déposé ce fichier dans la zone de Drag\&Drop.
\begin{quote}\begin{description}
\item[{Retourne}] \leavevmode
redirige la page web vers la Fenêtre choix fichier avec comme parmètre le chemin du fichier uploadé.

\end{description}\end{quote}

\end{fulllineitems}



\chapter{addQualitatives.py}
\label{\detokenize{addQualitatives:module-add.addQualitatives}}\label{\detokenize{addQualitatives::doc}}\label{\detokenize{addQualitatives:addqualitatives-py}}\index{add.addQualitatives (module)}

\section{Le module \sphinxstyleliteralintitle{Analyse de données qualitatives}}
\label{\detokenize{addQualitatives:le-module-analyse-de-donnees-qualitatives}}\index{nbElemListeCouple() (dans le module add.addQualitatives)}

\begin{fulllineitems}
\phantomsection\label{\detokenize{addQualitatives:add.addQualitatives.nbElemListeCouple}}\pysiglinewithargsret{\sphinxcode{add.addQualitatives.}\sphinxbfcode{nbElemListeCouple}}{\emph{listeEffectifs}}{}
Calcul l’effectif total des données de listeEffectifs.

La fonction se charge simple de calculer l’effectif total des données contenu dans la listeEffectifs ens sommant 
l’effectif de chaque tuple (couple{[}1{]}).
\begin{quote}\begin{description}
\item[{Paramètres}] \leavevmode
\sphinxstyleliteralstrong{listeEffectifs} \textendash{} liste de tuples (donnée, effectif)

\item[{Retourne}] \leavevmode
l’effectif total des données de listeEffectifs

\item[{Type retourné}] \leavevmode
\sphinxhref{https://docs.python.org/2/library/functions.html\#int}{int}

\end{description}\end{quote}

\end{fulllineitems}

\index{calculEffectifs() (dans le module add.addQualitatives)}

\begin{fulllineitems}
\phantomsection\label{\detokenize{addQualitatives:add.addQualitatives.calculEffectifs}}\pysiglinewithargsret{\sphinxcode{add.addQualitatives.}\sphinxbfcode{calculEffectifs}}{\emph{listeDonnees}}{}
Calcul l’effectifs pour chaque données contenu dans listeDonnees.

La fonction prend en entrée une liste contenant les données é analyser. Elle calculera les effectifs pour chaque valeur é 
l’aide d’un dictionnaire. Ce dictionnaire sera converti en liste de tuples et un tri sera effectué sur pour ordonner les
tuples.
\begin{quote}\begin{description}
\item[{Paramètres}] \leavevmode
\sphinxstyleliteralstrong{listeDonnees} \textendash{} liste contenant les données é analyser

\item[{Retourne}] \leavevmode
listeEffectifs: liste de tuples (donnée, effectif)

\item[{Type retourné}] \leavevmode
list

\end{description}\end{quote}

\end{fulllineitems}

\index{calculEffectifsCumules() (dans le module add.addQualitatives)}

\begin{fulllineitems}
\phantomsection\label{\detokenize{addQualitatives:add.addQualitatives.calculEffectifsCumules}}\pysiglinewithargsret{\sphinxcode{add.addQualitatives.}\sphinxbfcode{calculEffectifsCumules}}{\emph{listeEffectifs}}{}
Calcul l’effectifs cumulés avec l’aide de listeEffectifs.

La fonction prend en entrée la liste des effectifs. Elle calculera dans une nouvelle liste l’effectifs cumulés é partir 
de « listeEffectifs » en remplaéant l’effectif par l’effectif cumulés.
\begin{quote}\begin{description}
\item[{Paramètres}] \leavevmode
\sphinxstyleliteralstrong{listeEffectifs} \textendash{} liste de tuples (donnée, effectif)

\item[{Retourne}] \leavevmode
listeEffectifsCumules: liste de tuples (donnée, effectif cumulé)

\item[{Type retourné}] \leavevmode
list

\end{description}\end{quote}

\end{fulllineitems}

\index{calculFrequences() (dans le module add.addQualitatives)}

\begin{fulllineitems}
\phantomsection\label{\detokenize{addQualitatives:add.addQualitatives.calculFrequences}}\pysiglinewithargsret{\sphinxcode{add.addQualitatives.}\sphinxbfcode{calculFrequences}}{\emph{listeEffectifs}}{}
Calcul les fréquences d’apparitions des valeurs.

La fonction prend en entrée la liste des effectifs. Elle calculera dans une nouvelle liste la fréquence é partir 
de « listeEffectifs » en remplaéant l’effectif par la fréquence.
\begin{quote}\begin{description}
\item[{Paramètres}] \leavevmode
\sphinxstyleliteralstrong{listeEffectifs} \textendash{} liste de tuples (donnée, effectif)

\item[{Retourne}] \leavevmode
listeFrequences: liste de tuples (donnée, frequence)

\item[{Type retourné}] \leavevmode
list

\end{description}\end{quote}

\end{fulllineitems}

\index{calculFrequencesCumulees() (dans le module add.addQualitatives)}

\begin{fulllineitems}
\phantomsection\label{\detokenize{addQualitatives:add.addQualitatives.calculFrequencesCumulees}}\pysiglinewithargsret{\sphinxcode{add.addQualitatives.}\sphinxbfcode{calculFrequencesCumulees}}{\emph{listeFrequences}}{}
Calcul les fréquences cumulés pour une liste de fréquences.

La fonction prend en entrée la liste des frequences. Elle calculera dans une nouvelle liste la fréquence é partir 
de « listeFrequence » en remplaéant la fréquence par la fréquence cumulé.
\begin{quote}\begin{description}
\item[{Paramètres}] \leavevmode
\sphinxstyleliteralstrong{listeFrequences} \textendash{} liste de tuples (donnée, frequence)

\item[{Retourne}] \leavevmode
listeFrequences: liste de tuples (donnée, frequence cumulé)

\item[{Type retourné}] \leavevmode
list

\end{description}\end{quote}

\end{fulllineitems}

\index{infoSecteurs() (dans le module add.addQualitatives)}

\begin{fulllineitems}
\phantomsection\label{\detokenize{addQualitatives:add.addQualitatives.infoSecteurs}}\pysiglinewithargsret{\sphinxcode{add.addQualitatives.}\sphinxbfcode{infoSecteurs}}{\emph{listeFrequences}}{}
Stock dans un fichier JSON les informations nécessaire é la création d’un diagramme de secteurs.

La fonction prend en entrée le résultat du calcul des fréquences .Elle va créer un fichier .json pour y stocker (écrire)
les données nécessaires é la construction du diagramme en secteurs. Pour chaque couple (fréquence,valeur) elle va associer 
un angle compris entre 0éet 360é.
\begin{quote}\begin{description}
\item[{Paramètres}] \leavevmode
\sphinxstyleliteralstrong{listeFrequences} \textendash{} liste de tuples (donnée, fréquence)

\end{description}\end{quote}

\end{fulllineitems}

\index{infoHistogramme() (dans le module add.addQualitatives)}

\begin{fulllineitems}
\phantomsection\label{\detokenize{addQualitatives:add.addQualitatives.infoHistogramme}}\pysiglinewithargsret{\sphinxcode{add.addQualitatives.}\sphinxbfcode{infoHistogramme}}{\emph{listeEffectifs}}{}
Stock dans un fichier JSON les informations nécessaire é la création d’un histogramme.

La fonction prend en entrée le résultat du calcul des effectifs préalablement stocké dans une liste listeEffectifs. Elle 
va créer un fichier .json pour y stocker les données nécessaires é la construction de léhistogramme.
\begin{quote}\begin{description}
\item[{Paramètres}] \leavevmode
\sphinxstyleliteralstrong{listeEffectifs} \textendash{} liste de tuples (donnée, effectif)

\end{description}\end{quote}

\end{fulllineitems}



\chapter{addQuantitativesContinues.py}
\label{\detokenize{addQuantitativesContinues:module-add.addQuantitativesContinues}}\label{\detokenize{addQuantitativesContinues:addquantitativescontinues-py}}\label{\detokenize{addQuantitativesContinues::doc}}\index{add.addQuantitativesContinues (module)}

\section{Le module \sphinxstyleliteralintitle{Analyse de données quantitatives continues}}
\label{\detokenize{addQuantitativesContinues:le-module-analyse-de-donnees-quantitatives-continues}}\index{discretisation() (dans le module add.addQuantitativesContinues)}

\begin{fulllineitems}
\phantomsection\label{\detokenize{addQuantitativesContinues:add.addQuantitativesContinues.discretisation}}\pysiglinewithargsret{\sphinxcode{add.addQuantitativesContinues.}\sphinxbfcode{discretisation}}{\emph{nombreClasses}, \emph{donneesContinues}}{}
Discrétise des données continues du paramètre.

La fonction se charge de décomposer l’étendue {[}min ; max{]} de l’ensemble de données en \sphinxcode{nombreClasses} intervalles de même étendue.
Ensuite de remplacer les occurrences des données par l’intervalle auquel la donnée appartient.
\begin{quote}\begin{description}
\item[{Paramètres}] \leavevmode
\sphinxstyleliteralstrong{donneesContinues} \textendash{} liste de nombres flottants

\item[{Retourne}] \leavevmode
liste d’intervalles, et étendue discrétisée

\end{description}\end{quote}

\end{fulllineitems}

\index{calculNombreClasses() (dans le module add.addQuantitativesContinues)}

\begin{fulllineitems}
\phantomsection\label{\detokenize{addQuantitativesContinues:add.addQuantitativesContinues.calculNombreClasses}}\pysiglinewithargsret{\sphinxcode{add.addQuantitativesContinues.}\sphinxbfcode{calculNombreClasses}}{\emph{donneesContinues}}{}
Calcule le nombre de classes nécessaire à une discrétisation selon la règle de Sturges.
\begin{quote}\begin{description}
\item[{Type retourné}] \leavevmode
\sphinxhref{https://docs.python.org/2/library/functions.html\#int}{int}

\end{description}\end{quote}

\begin{sphinxadmonition}{warning}{Avertissement:}
Si la distribution n’est pas symétrique, le nombre de classes ne sera pas optimal.
\end{sphinxadmonition}

\end{fulllineitems}

\index{preparationIntervallesAnalyse() (dans le module add.addQuantitativesContinues)}

\begin{fulllineitems}
\phantomsection\label{\detokenize{addQuantitativesContinues:add.addQuantitativesContinues.preparationIntervallesAnalyse}}\pysiglinewithargsret{\sphinxcode{add.addQuantitativesContinues.}\sphinxbfcode{preparationIntervallesAnalyse}}{\emph{listeIntervalles}}{}
Prépare les données pour l’utilisation des éléments de calcul du module ADD quantitatives discrètes.

Pour effectuer les analyses descriptives dans le cas continu, la démarche est la même (sauf quantiles) que pour le cas discret.
On utilisera cependant comme données les centres des intervalles.
\begin{quote}\begin{description}
\item[{Paramètres}] \leavevmode
\sphinxstyleliteralstrong{listeIntervalles} \textendash{} liste issue de la discrétisation des valeurs.

\item[{Retourne}] \leavevmode
liste de flottants.

\end{description}\end{quote}

\end{fulllineitems}

\index{interpolationLineaire() (dans le module add.addQuantitativesContinues)}

\begin{fulllineitems}
\phantomsection\label{\detokenize{addQuantitativesContinues:add.addQuantitativesContinues.interpolationLineaire}}\pysiglinewithargsret{\sphinxcode{add.addQuantitativesContinues.}\sphinxbfcode{interpolationLineaire}}{\emph{p1}, \emph{p2}, \emph{y}}{}
Calcule l’abscisse par interpolation linéaire

Les points \sphinxcode{p1}, \sphinxcode{p2} nous permettent de définir une fonction linéaire f(x) = pente * x + ordonnée à l’origine (oo).
On retrouve ensuite l’abscisse du point d’ordonnée \sphinxcode{y} se trouvant sur la courbe de la fonction, x = (\sphinxcode{y} - oo) / pente.
\begin{quote}\begin{description}
\item[{Retourne}] \leavevmode
abscisse de l’ordonnée \sphinxcode{y} par rapport à la droite (\sphinxcode{p1}, \sphinxcode{p2})

\item[{Type retourné}] \leavevmode
\sphinxhref{https://docs.python.org/2/library/functions.html\#float}{float}

\end{description}\end{quote}

\end{fulllineitems}

\index{quantileContinu() (dans le module add.addQuantitativesContinues)}

\begin{fulllineitems}
\phantomsection\label{\detokenize{addQuantitativesContinues:add.addQuantitativesContinues.quantileContinu}}\pysiglinewithargsret{\sphinxcode{add.addQuantitativesContinues.}\sphinxbfcode{quantileContinu}}{\emph{ordre}, \emph{listeFrequencesCumulees}, \emph{intervalles}}{}
Calcule les quantiles d’ordre \sphinxcode{ordre} pour une analyse de données conitnues.

Le quantile discret nous permet de retrouver le centre de l’intervalle qui contient le vrai quantile.
Ensuite, à partir de l’intervalle et de l’ordre, on en déduit une valeur plus précise par interpolation linéaire.

La fonction linéaire est définie à l’aide des bornes de l’intervalle, on  a besoin de deux points :
L’ordonnée de la borne supérieure est la fréquence cumulée du centre de l’intervalle ( 1 si borne = max )
L’ordonnée de la borne inférieur est la fréquence cumulée du centre de l’intervalle précédent ( 0 si borne = min )
\begin{quote}\begin{description}
\item[{Retourne}] \leavevmode
le quantile d’ordre \sphinxcode{ordre}

\end{description}\end{quote}

\end{fulllineitems}

\index{infoDistributionCumulativeContinue() (dans le module add.addQuantitativesContinues)}

\begin{fulllineitems}
\phantomsection\label{\detokenize{addQuantitativesContinues:add.addQuantitativesContinues.infoDistributionCumulativeContinue}}\pysiglinewithargsret{\sphinxcode{add.addQuantitativesContinues.}\sphinxbfcode{infoDistributionCumulativeContinue}}{\emph{listeEffectifsCumules}, \emph{intervalles}}{}
Écriture dans le fichier distributionCumulative.json
\begin{description}
\item[{Format du fihier :}] \leavevmode
Début
\{
\begin{quote}

« x »: {[} liste des abscisses / bornes des intervalles {]},
« value »: {[} liste des ordonnées / effectifs cumulés {]}
\end{quote}

\}
Fin

\end{description}
\begin{quote}\begin{description}
\item[{Paramètres}] \leavevmode
\sphinxstyleliteralstrong{listeEffectifCumules} \textendash{} liste de couples (centre de l’intervalle, effectif cumulé).

\end{description}\end{quote}

\end{fulllineitems}



\chapter{addQuantitativesDiscretes.py}
\label{\detokenize{addQuantitativesDiscretes:addquantitativesdiscretes-py}}\label{\detokenize{addQuantitativesDiscretes:module-add.addQuantitativesDiscretes}}\label{\detokenize{addQuantitativesDiscretes::doc}}\index{add.addQuantitativesDiscretes (module)}

\section{Le module \sphinxstyleliteralintitle{Analyse de données quantitatives discrètes}}
\label{\detokenize{addQuantitativesDiscretes:le-module-analyse-de-donnees-quantitatives-discretes}}\index{moyenne() (dans le module add.addQuantitativesDiscretes)}

\begin{fulllineitems}
\phantomsection\label{\detokenize{addQuantitativesDiscretes:add.addQuantitativesDiscretes.moyenne}}\pysiglinewithargsret{\sphinxcode{add.addQuantitativesDiscretes.}\sphinxbfcode{moyenne}}{\emph{listeEffectifs}}{}
Calcule la moyenne arithmétique.
\begin{quote}\begin{description}
\item[{Paramètres}] \leavevmode
\sphinxstyleliteralstrong{listeEffectifs} \textendash{} liste de couples (valeur, occurences)

\item[{Retourne}] \leavevmode
moyenne arithmétique des valeurs de la liste

\end{description}\end{quote}

\end{fulllineitems}

\index{quantileDiscret() (dans le module add.addQuantitativesDiscretes)}

\begin{fulllineitems}
\phantomsection\label{\detokenize{addQuantitativesDiscretes:add.addQuantitativesDiscretes.quantileDiscret}}\pysiglinewithargsret{\sphinxcode{add.addQuantitativesDiscretes.}\sphinxbfcode{quantileDiscret}}{\emph{ordre}, \emph{listeFrequencesCumulees}}{}
Calcule le quantile d’ordre \sphinxcode{ordre}.
\begin{quote}\begin{description}
\item[{Paramètres}] \leavevmode\begin{itemize}
\item {} 
\sphinxstyleliteralstrong{ordre} \textendash{} Nombre flottant compris entre 0 et 1.

\item {} 
\sphinxstyleliteralstrong{listeFrequencesCumulees} \textendash{} liste de couples (valeur, frequence cumulee) triée selon les valeurs

\end{itemize}

\item[{Retourne}] \leavevmode
La première valeur telle que la fréquence cumulée correspondante soit supérieure ou égale à l’ordre.

\item[{Type retourné}] \leavevmode
\sphinxhref{https://docs.python.org/2/library/functions.html\#float}{float}

\end{description}\end{quote}

\begin{sphinxadmonition}{note}{Note:}
La médiane est le quantile d’ordre 1/2. Les quartiles sont les quantiles d’ordre 1/4 et 3/4.
\end{sphinxadmonition}

\end{fulllineitems}

\index{variance() (dans le module add.addQuantitativesDiscretes)}

\begin{fulllineitems}
\phantomsection\label{\detokenize{addQuantitativesDiscretes:add.addQuantitativesDiscretes.variance}}\pysiglinewithargsret{\sphinxcode{add.addQuantitativesDiscretes.}\sphinxbfcode{variance}}{\emph{listeEffectifs}}{}
Calcule la variance.

\end{fulllineitems}

\index{ecartType() (dans le module add.addQuantitativesDiscretes)}

\begin{fulllineitems}
\phantomsection\label{\detokenize{addQuantitativesDiscretes:add.addQuantitativesDiscretes.ecartType}}\pysiglinewithargsret{\sphinxcode{add.addQuantitativesDiscretes.}\sphinxbfcode{ecartType}}{\emph{variance}}{}
Calcule l’écart-type.

\end{fulllineitems}

\index{anomaliesTukey() (dans le module add.addQuantitativesDiscretes)}

\begin{fulllineitems}
\phantomsection\label{\detokenize{addQuantitativesDiscretes:add.addQuantitativesDiscretes.anomaliesTukey}}\pysiglinewithargsret{\sphinxcode{add.addQuantitativesDiscretes.}\sphinxbfcode{anomaliesTukey}}{\emph{listeEffectifs}}{}
Liste les valeurs aberrantes de la liste.

Une valeur est dite aberrante selon la règle de Tukey si elle n’appartient pas à un intervalle I définit tel que :
I = {[}Q1 - k * IQ ; Q3 + k * IQ{]} , k constante réelle Q1 et Q3 les quartiles, IQ l’écart inter-quartiles.

La constante k est choisie arbitrairement égale à 1,5. La valeur 1.5 est selon Tukey une valeur pragmatique, qui a une raison probabiliste.
Si une variable suit une distribution normale, alors la zone délimitée par la boîte et les moustaches devrait contenir 99,3 \% des observations.
\begin{quote}\begin{description}
\item[{Type retourné}] \leavevmode
list

\item[{Retourne}] \leavevmode
Collection contenant les données anormales pour la distribution des valeurs.

\end{description}\end{quote}

\end{fulllineitems}

\index{symetrie() (dans le module add.addQuantitativesDiscretes)}

\begin{fulllineitems}
\phantomsection\label{\detokenize{addQuantitativesDiscretes:add.addQuantitativesDiscretes.symetrie}}\pysiglinewithargsret{\sphinxcode{add.addQuantitativesDiscretes.}\sphinxbfcode{symetrie}}{\emph{listeEffectifs}}{}
Calcule le coefficient de symétrie de Fisher.

Si le coefficient est proche 0, la distribution est approximativement symétrique.
Si le coefficient est positif, la distribution est étalée sur la droite.
Si le coefficient est négatif, la distribution est étalée sur la gauche.
\begin{quote}\begin{description}
\item[{Type retourné}] \leavevmode
\sphinxhref{https://docs.python.org/2/library/functions.html\#float}{float}

\end{description}\end{quote}

\end{fulllineitems}

\index{aplatissement() (dans le module add.addQuantitativesDiscretes)}

\begin{fulllineitems}
\phantomsection\label{\detokenize{addQuantitativesDiscretes:add.addQuantitativesDiscretes.aplatissement}}\pysiglinewithargsret{\sphinxcode{add.addQuantitativesDiscretes.}\sphinxbfcode{aplatissement}}{\emph{listeEffectifs}}{}
Calcule le coefficient d’aplatissement de Fisher.

Si le coefficient est nul, la distribution suit une loi normale centrée réduite.
Si le coefficient est inférieur à 3, la distribution est aplatie.
Si le coefficient est supérieur à 3, les valeurs de la distribution est concentrée autour de la moyenne.
\begin{quote}\begin{description}
\item[{Type retourné}] \leavevmode
\sphinxhref{https://docs.python.org/2/library/functions.html\#float}{float}

\end{description}\end{quote}

\end{fulllineitems}

\index{infoDistributionDiscrete() (dans le module add.addQuantitativesDiscretes)}

\begin{fulllineitems}
\phantomsection\label{\detokenize{addQuantitativesDiscretes:add.addQuantitativesDiscretes.infoDistributionDiscrete}}\pysiglinewithargsret{\sphinxcode{add.addQuantitativesDiscretes.}\sphinxbfcode{infoDistributionDiscrete}}{\emph{listeEffectifs}}{}
Écriture dans le fichier distribution.js
\begin{description}
\item[{Format du fihier :}] \leavevmode
Début
\{
\begin{quote}

« x »: {[} liste des abscisses {]},
« value »: {[} liste des ordonnées / effectifs {]}
\end{quote}

\}
Fin

\end{description}
\begin{quote}\begin{description}
\item[{Paramètres}] \leavevmode
\sphinxstyleliteralstrong{listeEffectifs} \textendash{} liste de couples (valeur, effectif).

\end{description}\end{quote}

\end{fulllineitems}

\index{infoDistributionCumulativeDiscrete() (dans le module add.addQuantitativesDiscretes)}

\begin{fulllineitems}
\phantomsection\label{\detokenize{addQuantitativesDiscretes:add.addQuantitativesDiscretes.infoDistributionCumulativeDiscrete}}\pysiglinewithargsret{\sphinxcode{add.addQuantitativesDiscretes.}\sphinxbfcode{infoDistributionCumulativeDiscrete}}{\emph{listeEffectifsCumules}}{}
Écriture dans le fichier distributionCumulative.js
\begin{description}
\item[{Format du fihier :}] \leavevmode
Début
\{
\begin{quote}

« x »: {[} liste des abscisses {]},
« value »: {[} liste des ordonnées / effectifs cumulés {]}
\end{quote}

\}
Fin

\end{description}
\begin{quote}\begin{description}
\item[{Paramètres}] \leavevmode
\sphinxstyleliteralstrong{listeEffectifCumules} \textendash{} liste de couples (valeur, effectif cumulé).

\end{description}\end{quote}

\end{fulllineitems}

\index{infoBoiteTukey() (dans le module add.addQuantitativesDiscretes)}

\begin{fulllineitems}
\phantomsection\label{\detokenize{addQuantitativesDiscretes:add.addQuantitativesDiscretes.infoBoiteTukey}}\pysiglinewithargsret{\sphinxcode{add.addQuantitativesDiscretes.}\sphinxbfcode{infoBoiteTukey}}{\emph{listeEffectifs}}{}
Écriture dans le fichier boxplot.js
\begin{description}
\item[{Format du fihier :}] \leavevmode
Début
\{
\begin{quote}

« q1 »: premier quartile,
« median »: mediane,
« q3 »: troisième quartile,
« left »: extrémité gauche de la moustache (q1 - 1.5*(q3-q1)),
« right »: extrémité droite de la moustache (q3 + 1.5*(q3-q1)),
« outliers »: liste des anomalies statistiques
\end{quote}

\}
Fin

\end{description}

Informations utiles à la création d’une boîte à moustaches de Tukey
\begin{quote}\begin{description}
\item[{Paramètres}] \leavevmode
\sphinxstyleliteralstrong{listeEffectifs} \textendash{} liste de couples (valeur, effectif).

\end{description}\end{quote}

\end{fulllineitems}

\index{infoSerieTemporelle() (dans le module add.addQuantitativesDiscretes)}

\begin{fulllineitems}
\phantomsection\label{\detokenize{addQuantitativesDiscretes:add.addQuantitativesDiscretes.infoSerieTemporelle}}\pysiglinewithargsret{\sphinxcode{add.addQuantitativesDiscretes.}\sphinxbfcode{infoSerieTemporelle}}{\emph{listeSerieTemporelle}}{}
Écriture dans le fichier timeSeries.js
\begin{description}
\item[{Format du fihier :}] \leavevmode
Début
\{
\begin{quote}

« x »: {[} liste des Timestamp {]},
« value »: {[} liste des valeurs {]}
\end{quote}

\}
Fin

\end{description}
\begin{quote}\begin{description}
\item[{Paramètres}] \leavevmode
\sphinxstyleliteralstrong{listeSerieTemporelle} \textendash{} liste de couples (Timestamp, valeur), et un Timestamp est une chaîne de caractères.

\end{description}\end{quote}

\end{fulllineitems}



\chapter{verificationFormatFichier.py}
\label{\detokenize{verificationFormatFichier:module-chargement_des_donnees.verificationFormatFichier}}\label{\detokenize{verificationFormatFichier::doc}}\label{\detokenize{verificationFormatFichier:verificationformatfichier-py}}\index{chargement\_des\_donnees.verificationFormatFichier (module)}

\section{Le module \sphinxstyleliteralintitle{Vérification format fichier}}
\label{\detokenize{verificationFormatFichier:le-module-verification-format-fichier}}\index{verifOuverture() (dans le module chargement\_des\_donnees.verificationFormatFichier)}

\begin{fulllineitems}
\phantomsection\label{\detokenize{verificationFormatFichier:chargement_des_donnees.verificationFormatFichier.verifOuverture}}\pysiglinewithargsret{\sphinxcode{chargement\_des\_donnees.verificationFormatFichier.}\sphinxbfcode{verifOuverture}}{\emph{chemin}}{}
Fonctionnalité de vérification de l’existance du fichier pour l’ouverture
\begin{quote}\begin{description}
\item[{Paramètres}] \leavevmode
\sphinxstyleliteralstrong{chemin} (\sphinxhref{https://docs.python.org/2/library/functions.html\#str}{\sphinxstyleliteralemphasis{str}}) \textendash{} chemin du fichier

\item[{Retourne}] \leavevmode
entier 0 ou une description de l’erreur

\end{description}\end{quote}

\end{fulllineitems}

\index{verifExtension() (dans le module chargement\_des\_donnees.verificationFormatFichier)}

\begin{fulllineitems}
\phantomsection\label{\detokenize{verificationFormatFichier:chargement_des_donnees.verificationFormatFichier.verifExtension}}\pysiglinewithargsret{\sphinxcode{chargement\_des\_donnees.verificationFormatFichier.}\sphinxbfcode{verifExtension}}{\emph{chemin}}{}
Fonctionnalité de vérification de l’extension du fichier
\begin{quote}\begin{description}
\item[{Paramètres}] \leavevmode
\sphinxstyleliteralstrong{chemin} (\sphinxhref{https://docs.python.org/2/library/functions.html\#str}{\sphinxstyleliteralemphasis{str}}) \textendash{} chemin du fichier

\item[{Retourne}] \leavevmode
entier 0 ou une description de l’erreur

\end{description}\end{quote}

\end{fulllineitems}

\index{verifLecture() (dans le module chargement\_des\_donnees.verificationFormatFichier)}

\begin{fulllineitems}
\phantomsection\label{\detokenize{verificationFormatFichier:chargement_des_donnees.verificationFormatFichier.verifLecture}}\pysiglinewithargsret{\sphinxcode{chargement\_des\_donnees.verificationFormatFichier.}\sphinxbfcode{verifLecture}}{\emph{fichierCSV}}{}
Fonctionnalité de vérification de l’accès au contenu du fichier et de sa nature
\begin{quote}\begin{description}
\item[{Paramètres}] \leavevmode
\sphinxstyleliteralstrong{fichierCSV} (\sphinxstyleliteralemphasis{TextIoWrapper}) \textendash{} le fichier CSV ouvert

\item[{Retourne}] \leavevmode
entier 0 ou une description de l’erreur

\end{description}\end{quote}

\end{fulllineitems}

\index{ouvrir() (dans le module chargement\_des\_donnees.verificationFormatFichier)}

\begin{fulllineitems}
\phantomsection\label{\detokenize{verificationFormatFichier:chargement_des_donnees.verificationFormatFichier.ouvrir}}\pysiglinewithargsret{\sphinxcode{chargement\_des\_donnees.verificationFormatFichier.}\sphinxbfcode{ouvrir}}{\emph{chemin}}{}
Fonctionnalité principale d’ouverture du fichier CSV et de vérification
\begin{quote}\begin{description}
\item[{Paramètres}] \leavevmode
\sphinxstyleliteralstrong{chemin} (\sphinxhref{https://docs.python.org/2/library/functions.html\#str}{\sphinxstyleliteralemphasis{str}}) \textendash{} chemin du fichier

\item[{Retourne}] \leavevmode
le fichier CSV ouvert ou la description de l’erreur rencontrée lors de l’ouverture

\end{description}\end{quote}

\end{fulllineitems}



\chapter{analyseContenuFichier.py}
\label{\detokenize{analyseContenuFichier:analysecontenufichier-py}}\label{\detokenize{analyseContenuFichier::doc}}\label{\detokenize{analyseContenuFichier:module-chargement_des_donnees.analyseContenuFichier}}\index{chargement\_des\_donnees.analyseContenuFichier (module)}

\section{Le module \sphinxstyleliteralintitle{Analyse Contenu fichier}}
\label{\detokenize{analyseContenuFichier:le-module-analyse-contenu-fichier}}\index{lecture() (dans le module chargement\_des\_donnees.analyseContenuFichier)}

\begin{fulllineitems}
\phantomsection\label{\detokenize{analyseContenuFichier:chargement_des_donnees.analyseContenuFichier.lecture}}\pysiglinewithargsret{\sphinxcode{chargement\_des\_donnees.analyseContenuFichier.}\sphinxbfcode{lecture}}{\emph{fichierCSV}}{}
Fonction de lecture du contenu du fichier CSV ligne par ligne
\begin{quote}\begin{description}
\item[{Paramètres}] \leavevmode
\sphinxstyleliteralstrong{fichierCSV} (\sphinxstyleliteralemphasis{TextIoWrapper}) \textendash{} fichier CSV ouvert et vérifié

\item[{Retourne}] \leavevmode
liste dont chaque élément est une sous-liste contenant les données d’une ligne du fichier

\end{description}\end{quote}

\end{fulllineitems}

\index{typeDeDonnee() (dans le module chargement\_des\_donnees.analyseContenuFichier)}

\begin{fulllineitems}
\phantomsection\label{\detokenize{analyseContenuFichier:chargement_des_donnees.analyseContenuFichier.typeDeDonnee}}\pysiglinewithargsret{\sphinxcode{chargement\_des\_donnees.analyseContenuFichier.}\sphinxbfcode{typeDeDonnee}}{\emph{chaine}}{}
Fonction de detection du type de donnée depuis une chaine de carcteres
\begin{quote}\begin{description}
\item[{Paramètres}] \leavevmode
\sphinxstyleliteralstrong{fichierCSV} (\sphinxstyleliteralemphasis{TextIoWrapper}) \textendash{} fichier CSV

\item[{Retourne}] \leavevmode
liste dont chaque élément est une sous-liste contenant les données d’une ligne du fichier

\end{description}\end{quote}

\end{fulllineitems}

\index{removeDateSuffix() (dans le module chargement\_des\_donnees.analyseContenuFichier)}

\begin{fulllineitems}
\phantomsection\label{\detokenize{analyseContenuFichier:chargement_des_donnees.analyseContenuFichier.removeDateSuffix}}\pysiglinewithargsret{\sphinxcode{chargement\_des\_donnees.analyseContenuFichier.}\sphinxbfcode{removeDateSuffix}}{\emph{chaineDate}}{}
\end{fulllineitems}

\index{descriptionColonnes() (dans le module chargement\_des\_donnees.analyseContenuFichier)}

\begin{fulllineitems}
\phantomsection\label{\detokenize{analyseContenuFichier:chargement_des_donnees.analyseContenuFichier.descriptionColonnes}}\pysiglinewithargsret{\sphinxcode{chargement\_des\_donnees.analyseContenuFichier.}\sphinxbfcode{descriptionColonnes}}{\emph{lignesCSV}}{}
Fonction de description du nom, du type et des erreurs des colonnes du fichier CSV
\begin{quote}\begin{description}
\item[{Paramètres}] \leavevmode
\sphinxstyleliteralstrong{lignesCSV} (\sphinxstyleliteralemphasis{list}) \textendash{} lignes du fichier CSV

\item[{Retourne}] \leavevmode
dictionnaire de 3 sous-listes ayant pour clés : « nom », « type » et « erreurs »

\end{description}\end{quote}

\end{fulllineitems}

\index{analyseFichier() (dans le module chargement\_des\_donnees.analyseContenuFichier)}

\begin{fulllineitems}
\phantomsection\label{\detokenize{analyseContenuFichier:chargement_des_donnees.analyseContenuFichier.analyseFichier}}\pysiglinewithargsret{\sphinxcode{chargement\_des\_donnees.analyseContenuFichier.}\sphinxbfcode{analyseFichier}}{\emph{fichierCSV}}{}
Fonctionnalité principale d’analyse du contenu du fichier CSV ouvert
\begin{quote}\begin{description}
\item[{Paramètres}] \leavevmode
\sphinxstyleliteralstrong{fichierCSV} (\sphinxstyleliteralemphasis{TextIoWrapper}) \textendash{} le fichier CSV ouvert et vérifié

\item[{Retourne}] \leavevmode
une liste contenant les données du fichier et un dictionnaire décrivant ces données

\end{description}\end{quote}

\end{fulllineitems}



\renewcommand{\indexname}{Index des modules Python}
\begin{sphinxtheindex}
\def\bigletter#1{{\Large\sffamily#1}\nopagebreak\vspace{1mm}}
\bigletter{a}
\item {\sphinxstyleindexentry{add.addQualitatives}}\sphinxstyleindexpageref{addQualitatives:\detokenize{module-add.addQualitatives}}
\item {\sphinxstyleindexentry{add.addQuantitativesContinues}}\sphinxstyleindexpageref{addQuantitativesContinues:\detokenize{module-add.addQuantitativesContinues}}
\item {\sphinxstyleindexentry{add.addQuantitativesDiscretes}}\sphinxstyleindexpageref{addQuantitativesDiscretes:\detokenize{module-add.addQuantitativesDiscretes}}
\indexspace
\bigletter{c}
\item {\sphinxstyleindexentry{chargement\_des\_donnees.analyseContenuFichier}}\sphinxstyleindexpageref{analyseContenuFichier:\detokenize{module-chargement_des_donnees.analyseContenuFichier}}
\item {\sphinxstyleindexentry{chargement\_des\_donnees.verificationFormatFichier}}\sphinxstyleindexpageref{verificationFormatFichier:\detokenize{module-chargement_des_donnees.verificationFormatFichier}}
\indexspace
\bigletter{i}
\item {\sphinxstyleindexentry{interface\_web.choixFichier}}\sphinxstyleindexpageref{choixFichier:\detokenize{module-interface_web.choixFichier}}
\item {\sphinxstyleindexentry{interface\_web.gestionFlux}}\sphinxstyleindexpageref{gestionFlux:\detokenize{module-interface_web.gestionFlux}}
\end{sphinxtheindex}

\renewcommand{\indexname}{Index}
\printindex
\end{document}