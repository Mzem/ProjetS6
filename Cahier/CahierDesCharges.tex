\documentclass[a4paper,12pt]{article}
\usepackage[utf8]{inputenc}
\usepackage[T1]{fontenc}
\usepackage[french]{babel}
\usepackage[right=2.5cm, left=2.5cm]{geometry}
\usepackage[ddmmyyyy]{datetime}
\usepackage[table]{xcolor}
\usepackage{lmodern,mathptmx,changepage,titlesec,hyperref,listings,lstautogobble,graphicx,array,longtable,multirow,lipsum,tikz,shorttoc,enumitem}
\usetikzlibrary{arrows,automata}
\usetikzlibrary{positioning}

\renewcommand{\rmdefault}{\sfdefault} %Utilisation de la police sans-serif ("Computer Modern Sans") pour la police roman
\renewcommand{\ttdefault}{pcr} 	%Utilisation d'une police "CourrierNew" pour la police monospaced (pour faire un listing manuel)
\linespread{1.15}				%Interligne

%Utilisation de liens colorés en bleu et soulignés
\hypersetup{colorlinks=true, urlcolor=blue, urlbordercolor=blue, linkcolor=black, linkbordercolor=white}
\makeatletter \Hy@AtBeginDocument{\def\@pdfborder{0 0 1} \def\@pdfborderstyle{/S/U/W 1}}\makeatother

\titlespacing*{\section} {0cm}{7ex plus 1ex minus .2ex}{1.5ex plus .2ex}
\titlespacing*{\subsection} {0cm}{4.5ex plus 1ex minus .2ex}{1.5ex plus .2ex}
\titleformat*{\section}{\huge\bfseries}
\titleformat*{\subsection}{\Large\bfseries}
\titleformat*{\subsubsection}{\normalsize\bfseries}

\definecolor{darkgreen}{rgb}{0,0.8,0}
\definecolor{mygray}{rgb}{0.93,0.93,0.93}
\definecolor{mymauve}{rgb}{0.58,0,0.82}
\lstset{	
	basicstyle=\small\ttfamily,
	backgroundcolor=\color{mygray},
	breaklines=true,
	breakatwhitespace=true,
	postbreak=\raisebox{0ex}[0ex][0ex]{\ensuremath{\color{red}\hookrightarrow\space}},
	tabsize=3,
	frame=none,
	rulecolor=\color{black},
	keywordstyle=\color{blue}\bfseries,
	stringstyle=\color{orange},
	showstringspaces=false,
	commentstyle=\footnotesize\color{darkgreen},
	keepspaces=true,
	extendedchars=true,
	numbers=left,
	numberstyle=\tiny\color{lightgray},
	stepnumber=1,
	escapeinside={(@}{@)},
	autogobble=true,
	literate=
		{á}{{\'a}}1 {é}{{\'e}}1 {í}{{}}1 {ó}{{\'o}}1 {ú}{{\'u}}1
		{Á}{{\'A}}1 {É}{{\'E}}1 {Í}{{\'I}}1 {Ó}{{\'O}}1 {Ú}{{\'U}}1
		{à}{{\`a}}1 {è}{{\`e}}1 {ì}{{\`i}}1 {ò}{{\`o}}1 {ù}{{\`u}}1
		{À}{{\`A}}1 {È}{{\'E}}1 {Ì}{{\`I}}1 {Ò}{{\`O}}1 {Ù}{{\`U}}1
		{ä}{{\"a}}1 {ë}{{\"e}}1 {ï}{{\"i}}1 {ö}{{\"o}}1 {ü}{{\"u}}1
		{Ä}{{\"A}}1 {Ë}{{\"E}}1 {Ï}{{\"I}}1 {Ö}{{\"O}}1 {Ü}{{\"U}}1
		{â}{{\^a}}1 {ê}{{\^e}}1 {î}{{\^i}}1 {ô}{{\^o}}1 {û}{{\^u}}1
		{Â}{{\^A}}1 {Ê}{{\^E}}1 {Î}{{\^I}}1 {Ô}{{\^O}}1 {Û}{{\^U}}1
		{œ}{{\oe}}1 {Œ}{{\OE}}1 {æ}{{\ae}}1 {Æ}{{\AE}}1 {ß}{{\ss}}1
		{ç}{{\c c}}1 {Ç}{{\c C}}1 {ø}{{\o}}1 {å}{{\r a}}1 {Å}{{\r A}}1
		{€}{{e}}1 {£}{{\pounds}}1 {«}{{\guillemotleft}}1
		{»}{{\guillemotright}}1 {ñ}{{\~n}}1 {Ñ}{{\~N}}1 {¿}{{?`}}1
}

%Redéfinition de la taille de \Huge pour le titre du document
\makeatletter\renewcommand\Huge{\@setfontsize\Huge{37pt}{40}}\makeatother
\date{\today}

\title{\vspace{\fill}\textbf{\Huge Cahier des Charges}}
\author{Sonny Klotz - Jean-Didier Pailleux - Malek Zemni\vspace{2em}\\\textit{Interface de chargement, de contrôle}\\\textit{et d’analyse statistique des données}\\\textit{pour la constitution d’un graphe de flux}\vspace{2em}}

\begin{document}
\maketitle\vspace{\fill}
\newpage
\tableofcontents
\newpage

	\section{Motivations du projet}
		\subsection{But du projet}
			\subsubsection{Contexte du projet}
			De nos jours, les masses de données collectées sont de plus en plus importantes. L'objectif principal de cette collecte de données est d'en extraire une valeur ajoutée. Or, ces données à l'état brut sont difficilement exploitables dû à leur volume et à leur complexité.\\
			Notre produit correspond au travail indispensable d'analyse de ces données, afin de faciliter leur exploitation.
			\subsubsection{Objectif du projet}
			Ce projet a pour but de fournir aux utilisateurs une application qui se chargera d'une part de structurer les données, les analyser et les visualiser, et d'autre part de préparer ces données pour un chargement via des API.
		
		
		\subsection{Parties prenantes}
			\subsubsection{Maître d'ouvrage}
			Notre interface de contrôle, de chargement, et d'analyse de données est développée pour DCbrain.\\
			Le projet a été lancé en collabaration avec l'UVSQ.
			
			\subsubsection{Client}
			DCbrain est également l'entreprise qui va bénéficier des paquets finaux après leur développement.
			
			\subsubsection{Autre partie prenante}
			Les industriels clients de DCbrain sont des parties prenantes indirectes. Notre travail doit pouvoir être utilisé par DCbrain pour renforcer leur application.
			
		\subsection{Utilisateurs du produit}
		En premier lieu, l’application va servir aux membres de DCbrain étant donné que leurs clients industriels (Total, ERDF,…) leur fournissent les fichiers CSV, afin qu’ils puissent appliquer des analyses descriptives sur les données dans le but de repérer des anomalies sur les réseaux de ces derniers. Puis en second lieu, il se pourrait que DCbrain veuille déployer l'application pour ses clients, dans ce cas les membres de ces organismes deviendront des utilisateurs.\\
		
		On peut supposer l'absence de restrictions d'utilisations, étant donné l'absence d'exigence de la part du maître d'œuvre sur ce sujet.

	\section{Contraintes sur le Projet}
		\subsection{Contraintes imposées}
			\subsubsection{Contraintes sur la conception :}
				\begin{center}\begin{longtable}{|>{\centering}m{3cm}|>{\raggedright\arraybackslash}m{10cm}|}			
				\hline \multicolumn{1}{|c}{\textbf{Contrainte}} & \multicolumn{1}{|c|}{\textbf{Fiche}} \\
				\hline 	1. Le produit doit fournir une application web &
						\begin{description}[style=unboxed,leftmargin=0.2cm]
						\item{\textbf{Description :}} une "applet" fonctionnant sur un navigateur web.
						\item{\textbf{Justification :}} assure une très grande portabilité et fournit à l'utilisateur une interface interactive.
						\item{\textbf{Critère de satisfaction :}} on peut lancer l'application sur un navigateur web.
						\end{description}\\
				\hline 2. Le produit doit être développé avec un langage de programmation compatible avec l'analyse de données &
						\begin{description}[style=unboxed,leftmargin=0.2cm]
						\item{\textbf{Description :}} le langage de programmation choisi doit inclure une bibliothèque qui permet d'analyser les données (taches de data mining et de machine Learning).
						\item{\textbf{Justification :}} permettre une analyse de données la plus efficace possible.
						\item{\textbf{Critère de satisfaction :}} on peut analyser les données de manière efficace.
						\end{description}\\
				\hline
				\end{longtable}\vspace{1em}\end{center}
				
				\paragraph{Langages de programmation utilisés :} les contraintes 1 et 2 nous permettent de nous fixer sur le choix du langage : \textbf{\textit{Java EE}}. D'une part, ce langage est orienté pour le développement d'applications web robustes et distribuées, déployées et exécutées sur un serveur d'applications. D'autre part, ce langage qui est basé sur Java est doté de plusieurs modules d'analyse de données, dont l'outil \textbf{\textit{Weka}} par exemple.\\
				De plus, l'applet Java doit être intégrée dans une page web pour être exécutée : on va donc avoir recours à des langages de balisage comme \textbf{\textit{HTML}} pour la présentation et \textbf{\textit{CSS}} pour la mise en forme.
				
			\subsubsection{Environnement de fonctionnement :} \colorbox{yellow}{J'enlève cette parie ?}
			\subsubsection{Applications partenaires :}
			\colorbox{yellow}{Je mets API en sortie comme exigence ?}
			Le produit doit prendre en compte de l'environnement de sortie adéquat pour la production des API, c'est à dire que cette API doit etre compatible avec l'outil DCbrain de visualisation des graphes de réseaux physiques.
		
		\subsection{Glossaire et conventions de dénomination}
		
		\subsection{Faits et hypothèses déterminants}
			\subsubsection{Facteurs influençant le produit, mais qui ne sont pas des contraintes imposées sur les exigences :}
			\subsubsection{Hypothèses que l’équipe fait sur le projet :}
		
	\section{Exigences fonctionnelles}
		{\color{red}
		Contexte du projet (diagramme de contexte métier) :\\	 
		Ce diagramme a pour objet d’identifier les flux d’informations entre les différents acteurs (prescripteurs, pharmaciens, préparateurs, infirmières) et le système d’information.
		Il s’agit là d’une description des frontières du système d’information, dont le périmètre peut être plus large que le système informatique que l’on veut mettre en place.
		}

		\subsection{Périmètre de l'œuvre}
		
		\subsection{Périmètre de l'ouvrage}

		\subsection{Exigences fonctionnelles et exigences sur les données}	
		{\color{red}
		Données: Spécification des principaux éléments ou objets métiers ou entités ou classes relatifs au système. Cela peut prendre la forme d’un premier jet de modèle des données, un modèle objet ou un modèle du domaine. Il peut aussi suffire de remplir correctement le glossaire du paragraphe 5. Rédiger un glossaire et faire un diagramme de donnée sont deux façons de modéliser des objets métiers. Il y en a bien d’autres.\\
		}
	
	\section{Exigences non fonctionnelles}
		\subsection{Interface utilisateur du projet}
		
		\subsubsection{Exigences d’apparence} 
		Le produit devra adopter une apparence simple, agréable et devra être facile à comprendre dès sa première prise en main.
		Le maitre d’œuvre a émis le souhait d’avoir de l’Anglais comme langue utilisée pour l’affichage textuel.

		\subsubsection{Exigences de style}
		L’application s’adresse à un public professionnel, alors l’une des exigences concernant le style de l’interface est qu’il devra opter pour un style classique, au goût du jour et professionnel. L’affichage des résultats et autre devra susciter la confiance auprès des utilisateurs.

		\subsection{Utilisabilité}
		Le produit devra être simple d’utilisation et facile à comprendre dès sa première prise en main, afin d’éviter une formation concernant la manipulation du produit pour les futurs utilisateurs.
		Il aidera également l’utilisateur à analyser une quantité importante de données.

		\subsection{Exigences de performance}
		Les exigences de performance du futur produit concerne concerne la latence acceptable que devra avoir l'application. Ici la durée de réponse de l’affichage d’un CSV, ou d’une requête concernant l’analyse de données venant de l’utilisateur ne devra pas être trop longue, le client à suggérer que cette attente devra être de l’ordre de la minute maximum. En ce qui concerne le temps de passage d'une fenêtre à l'autre (Choit de(s) colonne(s) pour l'exécution d'une analyse statistique, affichage de la fenêtre d'erreurs, lors du choit du méthode d’affichage des résultats,..), l’application devra répondre de manière fluide. 

		\subsection{Fonctionnement du projet}
			\subsubsection{Principales fonctionnalités}
			Tout d’abord, l’application devra avoir la fonction majeur de posséder une interface afin de permettre le chargement d’un fichier au format CSV. Après avoir sélectionné un fichier, l’application pourra afficher le contenu de ce fichier sous forme de tableau et les erreurs dans le contenu d'une ligne dans une nouvelle fenêtre. Si le fichier contient une quantité importante de données (Supérieur à 1000), l’affichage ne se fera que sur les 1000 premières données.
			Il n’y aura pas d’importance à déterminer le type de colonne car il y aura un format prédéfinis pour le CSV.\newline

			Ensuite, il y aura la possibilité de faire une analyse descriptive sur les colonnes du fichier. Avec une analyse qualitative pour obtenir les informations synthétiques tel que la Médiane et les Quantiles mais également les anomalies présente sur certaines valeurs. Puis une analyse quantitative pour obtenir par exemple la distribution des valeurs.\newline

			Enfin, l’application aura la fonctionnalité de récupérer les résultats de ces analyses sous forme d’API afin de permet à DCbrain de connecter ces nouvelles données à leur base de donnes pour insérer de nouvelles informations dans leurs graphe de flux.

			\subsubsection{Précision et exactitude}
			Lors de l'analyse des données, les résultats sur la Variance et la Moyenne doivent être calculée de façon très précise car elles vont servir de base dans d’autres calculs statistiques. Les autres valeurs calculées peuvent être les plus précises possible mais ces valeurs vont servir pour l'interprétation afin mieux comprendre ce qui se passe sur le réseau.

		\subsection{Maintenabilité du projet}
		Le produit doit pouvoir être maintenu par ses utilisateurs finaux ou par des développeurs qui ne sont pas les développeurs d’origines, dans le cas où DCbrain souhaiterait ajouter de nouvelles méthodes d’analyses descriptives ou de nouveaux procédés pour afficher les résultats de manière graphique afin de satisfaire les exigences de leurs clients.\newline
		
		Le produit devra permettre l'insertion d'éventuels API supplémentaires, tel que l'analyse Multidimensionnel qui utilisera l'API de l'analyse Unidimensionnel dans le but de fournir des descriptions statiques plus poussé pour obtenir une meilleur vue d'ensemble sur les données collectées. Mais encore l'insertion de technique d'analyse de Graphe afin d'obtenir des informations sur l'état d'un graphe.
		
		\subsection{Sécurité} \colorbox{yellow}{c pas bon}
		L’application ne sera pas connectée à un réseau, la question de la sécurité du système contre des attaques extérieures ne devrait sûrement pas se poser. Mais le produit doit être programmé de telle sorte qu’il n’endommage pas le système hébergeant l’application.
		
	\section{Autres aspects du projet}
		\subsection{Question ouvertes}
		
		\subsection{Solution étagère}
		
		\subsection{Problèmes nouveaux}
		
		\subsection{Tâche à faire}
		
		\subsection{Contrôle de la finalisation}
			hello
		\subsection{Risques liés au projet}
		
		\subsection{Estimation des coûts}
		
		\subsection{Documentation utilisateur et formation}
		
		\subsection{Question mise en attente}
		
		\subsection{Idées de solutions}
		

\end{document}
