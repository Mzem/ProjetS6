\input{../tex/preambule}

\title{\vspace{\fill}\textbf{\Huge Spécifications}}
\author{
	Sonny Klotz - Jean-Didier Pailleux - Malek Zemni
	\vspace{2em}\\
	\textit{Interface de chargement, de contrôle}\\\textit{et d’analyse statistique des données}\\\textit{pour la constitution d’un graphe de flux}
	\vspace{2em}
}

\begin{document}
\pagenumbering{gobble}\clearpage
\maketitle\vspace{8em}
\begin{center}\includegraphics[scale=0.7]{../Cahier/logo.png}\end{center}
\begin{flushright}Module \textit{Projet}\end{flushright}
\newpage
\tableofcontents
\newpage\clearpage\pagenumbering{arabic}

	\section*{Introduction}
		Ce document va décrire l'ensemble des exigence fonctionnelles que doit satisfaire notre produit final, c'est-à-dire les différentes fonctionnalités que notre application va fournir. Ces fonctionnalités seront présentées selon les modules de l'organigramme établi dans le cahier des charges. Ces modules eux-mêmes seront regroupés en packages.\\
		Ce document va donc décrire, pour chaque package de l'organigramme, les fonctionnalités de ces modules : d'abord ceux du package de chargement des données, ensuite ceux du package d'analyse descriptive des données et enfin ceux du package de l'interface web.
		
	\section{Package Chargement des données}
	Ce package va être livré au client pour une intégration externe. On pourra donc parler de API.
		
		\subsection{Module Vérification format fichier}
			Ce module va vérifier le format du fichier fourni en entrée en 3 points. Il aura donc 3 fonctionnalités :
			\begin{itemize}
				\item Vérification de l'ouverture du fichier 
				\begin{lstlisting}[language=Python]
					def verifOuverture(chemin):
				\end{lstlisting}
			\end{itemize}
		
		
		\subsection{Module Analyse contenu fichier}
	
	
	\section{Package Analyse descriptive des données}
	Ce package va être livré au client pour une intégration externe. On pourra donc parler de API.
	
	\section{Package Interface web}
	
	
	\section*{Conclusion}
		
		
\end{document}
