\input{../tex/preambule}

\title{\vspace{\fill}\textbf{\Huge Spécifications}}
\author{
	Sonny Klotz - Jean-Didier Pailleux - Malek Zemni
	\vspace{2em}\\
	\textit{Interface de chargement, de contrôle}\\\textit{et d’analyse statistique des données}\\\textit{pour la constitution d’un graphe de flux}
	\vspace{2em}
}

\begin{document}
\pagenumbering{gobble}\clearpage
\maketitle\vspace{8em}
\begin{center}\includegraphics[scale=0.7]{../Cahier/logo.png}\end{center}
\begin{flushright}Module \textit{Projet}\end{flushright}
\newpage
\tableofcontents
\newpage\clearpage\pagenumbering{arabic}

	\section*{Introduction}
		Ce document va décrire l'ensemble des exigences fonctionnelles que doit satisfaire notre produit final, c'est-à-dire les différentes fonctionnalités que notre application va fournir. Cette description va prendre en compte les caractéristiques des outils de développement choisis.
		\paragraph{}Notre outil, Python, est un langage de programmation hybride. On utilisera d'une part la programmation fonctionnelle pour les calculs, et d'autre part la programmation objet pour le développement des interfaces graphiques. Dans Python, le type de données n'est connu qu'à l'exécution (typage dynamique), par conséquent, ces types ne seront pas indiqués dans les signatures des fonctions et les structures des classes. Ils seront précisés dans des paragraphes explicatifs.
		\paragraph{}Pour les parties qui s'appuient sur une interaction avec l'utilisateur, notre démarche de description des fonctionnalités va essentiellement prendre en compte l'\textit{expérience utilisateur}\footnote{http://uxdesign.com/ux-defined}. Cette description sera donc axée sur la qualification du résultat et du ressenti de l'utilisateur lors de la manipulation de l'interface fournie (une illustration à l'aide de croquis), plutôt que sur les points techniques de l'application (fonctions et classes).
		\paragraph{}Les fonctionnalités de notre application seront présentées selon les modules de l'organigramme établi dans le cahier des charges. Ces modules eux-mêmes seront regroupés en packages. Ce document va donc décrire, pour chaque package de l'organigramme, les fonctionnalités de ses modules : d'abord ceux du package de chargement des données, puis ceux du package d'analyse descriptive des données et ensuite ceux du package de l'interface web. Les types Python utilisés seront décrits dans la dernière partie, le glossaire des types.
		
	\section{Package Chargement des données}
	Ce package est composé de deux modules qui ont pour fonction de traiter le fichier de données fourni : une vérification de son format et une analyse de son contenu. On pourra aussi parler de API puisque ce package peut être éventuellement livré en sortie. 
		
		\subsection{Module Vérification format fichier}
			Ce module va vérifier le format du fichier de données fourni en entrée en 3 points. Il aura donc 3 fonctionnalités :
			\begin{enumerate}
				\vspace{1em}\item Fonctionnalité de vérification de l'ouverture du fichier :
					\begin{lstlisting}
						verifOuverture(fichierCSV)
					\end{lstlisting}
					\underline{Paramètres :}
						\begin{description}[style=unboxed,leftmargin=0.2cm]
							\item\lstinline!fichierCSV! : \lstinline!TextIoWrapper! - représente le fichier CSV fourni.
						\end{description}
					\underline{Retour :} variable de type booléen.\\
					\underline{Description :} cette fonction prend en entrée le fichier CSV ouvert. Elle vérifie que le paramètre \lstinline!fichierCSV! contient bien des informations représentant un fichier quelconque, et renvoie un booléen vrai si c'est le cas, faux sinon.
				\vspace{1em}\item Fonctionnalité de vérification de l'extension du fichier ouvert :
					\begin{lstlisting}
						verifExtenstion(fichierCSV)
					\end{lstlisting}
					\underline{Paramètres :}
						\begin{description}[style=unboxed,leftmargin=0.2cm]
							\item\lstinline!fichierCSV! : \lstinline!TextIoWrapper! - représente le fichier CSV fourni.
						\end{description}
					\underline{Retour :} variable de type booléen.\\
					\underline{Description :} cette fonction prend en entrée le fichier CSV ouvert. Elle vérifie que l'information décrivant l'extension du fichier dans le paramètre \lstinline!fichierCSV! correspond bien au format CSV, et renvoie un booléen vrai si c'est le cas, faux sinon.
				\vspace{1em}\item Fonctionnalité de vérification de l'accessibilité en lecture du fichier :
					\begin{lstlisting}
						verifLecture(fichierCSV)
					\end{lstlisting}
					\underline{Paramètres :}
						\begin{description}[style=unboxed,leftmargin=0.2cm]
							\item\lstinline!fichierCSV! : \lstinline!TextIoWrapper! - représente le fichier CSV fourni.
						\end{description}
					\underline{Retour :} variable de type booléen.\\
					\underline{Description :} cette fonction prend en entrée le fichier CSV ouvert. Elle vérifie que le paramètre \lstinline!fichierCSV! représentant le fichier possède bien la propriété d'accès en lecture. Un test de lecture sera aussi effectué sur le fichier. La fonction renvoie un booléen vrai si l'accès en lecture est permis, faux sinon.
			\end{enumerate}
		
		
		\subsection{Module Analyse contenu fichier}
			Ce module va analyser le contenu du fichier fourni en lisant, d'une part, les données de ce fichier une à une et d'autre part, en repérant les données erronées. Il aura donc 2 fonctionnalités :
			\begin{enumerate}
				\vspace{1em}\item Fonctionnalité de lecture du contenu du fichier CSV :
					\begin{lstlisting}
						lecture(fichierCSV)
					\end{lstlisting}
					\underline{Paramètres :}
						\begin{description}[style=unboxed,leftmargin=0.2cm]
							\item\lstinline!fichierCSV! : \lstinline!TextIoWrapper! - représente le fichier CSV fourni.
						\end{description}
					\underline{Retour :} structure contenant les données du fichier, le nombre de lignes et le nombre de colonnes (connus à partir de la taille de la structure). \\
					\underline{Description :} cette fonction prend en entrée le fichier CSV ouvert. Elle crée une structure pour y sauvegarder le contenu du fichier représenté par le paramètre \lstinline!fichierCSV!. On lit ligne par ligne des caractères du fichier. A chaque fois qu’on détecte un caractère de séparation (une virgule, un point-virgule ou une tabulation), on stocke les caractères lus (la donnée) dans la structure.
			\end{enumerate}
	
	\section{Package Analyse descriptive des données}
	Ce package va être livré au client pour une intégration externe. On pourra donc parler de API.
	
		\subsection{Module Analyse de données qualitatives}
	Ce module va effectuer les calculs d'effectifs, d'effectifs cumulés, de fréquences et de fréquences cumulés. Ce module s'occupera également de fournir les informations nécessaire pour la construction d'un diagramme de secteur et d'un histogramme. Il aura donc 6 fonctionnalités :
	
		\begin{enumerate}
				\vspace{1em}\item Fonctionnalité pour calculer l'effectif :
					\begin{lstlisting}
						calculEffectif(donneeQualitative)
					\end{lstlisting}
					\underline{Paramètres :} Un ensemble \lstinline!donneeQualitative! - structure contenant des données de type qualitative fichier, le nombre de lignes et le nombre de colonnes\\
					\underline{Retour :} Une liste contenant les effectifs des données de la colonne.\\
					\underline{Description :} cette fonction prendra en paramètre la structure contenant les données du fichier. Elle calculera les effectifs pour chaque valeurs
					
				\vspace{1em}\item Fonctionnalité pour calculer l'effectif cumulé:
					\begin{lstlisting}
						calculEffectifCumule(listeEffectif)
					\end{lstlisting}
					\underline{Paramètres :} \lstinline!listeEffectif! est de type liste, et correspond au résultat de la fonction \lstinline!calculEffectif(structure1)!\\
					\underline{Description :} 
					
				\vspace{1em}\item Fonctionnalité pour calculer la fréquence :
					\begin{lstlisting}
						calculFrequence(donneeQualitative)
					\end{lstlisting}
					\underline{Paramètres :} un objet \lstinline!strucutre1! - structure contenant les données du fichier, le nombre de lignes et le nombre de colonnes\\\\
					\underline{Retour :} Une liste contenant la fréquence des données de la colonne.\\
					\underline{Description :} 
					
				\vspace{1em}\item Fonctionnalité pour calculer la fréquence cumulé :
					\begin{lstlisting}
						calculFrequenceCumule(listeFrequence)
					\end{lstlisting}
					\underline{Paramètres :} \lstinline!listeFrequence! est de type liste, et correspond au résultat de la fonction \lstinline!calculFrequence(structure1)! \\ \\
					\underline{Retour :} Une liste \\
					\underline{Description :} 
					
				\vspace{1em}\item Fonctionnalité pour les données du diagramme en secteur :
					\begin{lstlisting}
						infoSecteur(something)
					\end{lstlisting}
					\underline{Paramètres :} \\
					\underline{Retour :} Une liste contenant les informations nécessaire pour la construction d'un diagramme de secteur.\\
					\underline{Description :} 
					
				\vspace{1em}\item Fonctionnalité pour les données de l'histogramme :
					\begin{lstlisting}
						infoHistogramme(something)
					\end{lstlisting}
					\underline{Paramètres :} \\
					\underline{Retour :}  Une liste contenant les informations nécessaire pour la construction d'un histogramme.\\
					\underline{Description :} 
			\end{enumerate}
		
		\subsection{Module Analyse de données quantitatives discrètes}
			
		\subsection{Module Analyse de données quantitatives continues}
		
		
	\section{Package Interface web}
		
		\subsection{Module Gestion des flux}
		
		\subsection{Module Fenêtre choix fichier}
		Ce module est la fenêtre qui va permettre à l'utilisateur d'aller charger le fichier CSV. Il y aura 3 fonctionnalités :

			\begin{enumerate}
				\vspace{1em}\item Fonctionnalité pour ouvrir le fichier avec le Système de Gestion de fichier :
					\begin{lstlisting}
						FileWithSGF()
					\end{lstlisting}
					\underline{Paramètres :} Ne demande aucun paramètre en entrée.\\
					\underline{Retour :} chaîne de caractères correspondant au chemin de l'emplacement du fichier\\
					\underline{Description :} cette fonction ne prend pas de paramètre en entrée. Elle permet de récupérer le chemin de l'emplacement du fichier CSV avec la technique du Drag\&Drop. Elle renvoie donc une chaîne de caractères correspondant à ce chemin.
					
				\vspace{1em}\item Fonctionnalité pour ouvrir le fichier avec la technique du Drag\&Drop.
					\begin{lstlisting}
						FileWithDragDrop()
					\end{lstlisting}
					\underline{Paramètres :} Ne demande aucun paramètre en entrée.\\
					\underline{Retour :} chaîne de caractères correspondant au chemin de l'emplacement du fichier\\
					\underline{Description :} cette fonction ne prend pas de paramètre en entrée. Elle permet de récupérer le chemin de l'emplacement du fichier CSV en parcourant le système de gestion de fichiers. Elle renvoie donc une chaîne de caractères correspondant à ce chemin.
					
				\vspace{1em}\item Fonctionnalité qui va ouvrir en lecture le fichier CSV.
					\begin{lstlisting}
						OpenFile(cheminFichier)
					\end{lstlisting}
					\underline{Paramètres :}
						\begin{description}[style=unboxed,leftmargin=0.2cm]
							\item\lstinline!cheminFichier! : chaîne de caractères - chemin du fichier CSV fourni.
						\end{description}
					\underline{Retour :} variable de type booléen.\\
					\underline{Description :} cette fonction prend en entrée le chemin du fichier CSV. Elle va faire appel aux fonctionnalités du module "Vérification format fichier" pour savoir si le fichier peut-être ouvert ou non. Dans ce cas la fonction renvoie un booléen à vrai si le fichier à bien été ouvert, faux sinon.
			\end{enumerate}
		
		\subsection{Module Fenêtre rôle et choix colonne}

		\subsection{Module Fenêtre résultats ADD, footnote{ADD : analyse descriptive de données}}
	
	\section{Glossaire des types}
	
	Dans ce glossaire, on va préciser la définition en Python des types des données qu'on a utilisé dans la description des fonctionnalités.
	\begin{itemize}
		\item \lstinline!TextIoWrapper! : classe représentant les flux de texte bufferisés (entrées/sorties de texte avec une sauvegarde dans une mémoire tampon). Elle permet donc de représenter les fichiers de texte brut en particulier. Elle définit des méthodes de manipulation de ces flux.
	\end{itemize}
	
	\section*{Conclusion}
		Bilan : rappel ce qu'on a fait et ce que ça a apporté\\
		Ouvertures :\\
			- Mise en place des tests d'acceptations (comment on va mesurer, il faut définir une mesure et après la calculer)\\
			- Difficultés\\
			- Limites des spécifications dans notre cas : nécessité de tester une 1ere version de l'appli pour bien definir l'experience utilisateur\\
		Dernière phrase positive :
			- Decouverte nouveaux outils, nouvelle demarche pour le dev : devops, ux design, balsamiq
		
\end{document}
