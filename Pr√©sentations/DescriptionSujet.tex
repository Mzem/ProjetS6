\documentclass{beamer}

\usepackage[utf8]{inputenc}
\usepackage[T1]{fontenc}
\usepackage[french]{babel}
\usepackage[ddmmyyyy]{datetime}

\usetheme{Warsaw}
\useinnertheme{rectangles}
\setbeamerfont{headline}{size=\large}
\setbeamerfont{frametitle}{size=\normalsize}

%Plan/Sommaire automatique avant chaque section
\AtBeginSection[]{
  \begin{frame}
  \frametitle{Plan}
  \tableofcontents[currentsection]
  \end{frame}
}

\author{Sonny Klotz - Jean-Didier Pailleux - Malek Zemni}
\institute{UVSQ}
\date{\today}
\title{Documentation sur le sujet}

\begin{document}

	\begin{frame}
		\titlepage
	\end{frame}
	
	\section{Introduction}
	\begin{frame}
		DCbrain est un logiciel qui permet de visualiser ce qui ce passe sur les \textbf{réseaux physiques (de fluide)} pour pouvoir trouver/prédire les problèmes et optimiser ces réseaux.\\~\\
		\pause
		Cette visualisation provient de données collectées à partir des réseaux physiques (à l'aide de mesures, de capteurs IOT, etc.) puis analysées grâce aux technologies du \textbf{Big Data}.\\~\\
		\pause
		
		\begin{block}{Réseaux physiques}
		Les réseaux qu'on va traiter dans le cadre de ce logiciel sont des réseaux industriels physiques (des réseaux de fluide, de distribution). Il s'agit des réseaux industriels tels les réseaux de distribution pétrolière, gazière, électrique, etc.\footnotemark
		\end{block}
		
		\footnotetext{\href{https://fr.wikipedia.org/wiki/R\%C3\%A9seau\_de\_distribution\_(fluides)}{https://fr.wikipedia.org/wiki/R\%C3\%A9seau\_de\_distribution\_(fluides)}}
	\end{frame}
	
	\begin{frame} 
	\begin{block}{Big Data}
	Décrit des ensembles de très gros volumes de données, à la fois structurées, semi-structurées ou non structurées, qui peuvent être traitées et exploitées dans le but d’en tirer des informations intelligibles et pertinentes.
	\end{block}\\~\\
	\pause	
	Exemple de Sources: 
	\begin{itemize}
		\item Capteurs utilisés pour collecter les informations climatiques, de trafic, consommation (Smart cities, Internet des Objets)
	 	\item Messages sur les Réseaux sociaux 
	 	\item Enregistrements transactionnels d’achat en ligne 
		\item Signaux GPS de téléphones mobile
	\end{itemize}
	\end{frame}
	
	\begin{frame}
	\frametitle{Problème}
	\end{frame}
	
	\section{Analyse descriptive de données}
	\subsection{sous-section1}
	\begin{frame}
		blabla sonny
	\end{frame}
	\subsection{sous-section2}
	\begin{frame}
		blabla sonny 2
	\end{frame}
	
	\section{Big Data et Machine Learning}
	\subsection{Big Data}
	\begin{frame}
	\begin{block}{Big Data}
	Le Big Data fait référence à la masse de données collectée.
	On considère du Big Data quand le traitement devient trop long pour une seule machine.
	\end{block}
			
		Le Big Data est caractérisé par les  3V: 
		\begin{itemize}
			\item le Volume de données considérable à traiter.
			\item la Variété de ces données qui peuvent être brutes, non structurées ou semi-structurées
			\item la Vélocité qui désigne le fait que ces données sont produites, récoltées et analysées en temps réel.
		\end{itemize}
	Les traitements de cette quantité importante de données est massivement "parallélisé" avec MapReduce/Hadoop. 
		
	\end{frame}
	\subsection{Machine Learning}
	\begin{frame}
 
\begin{block}{Machine Learning est:}
	\begin{itemize}	
	 \item Une discipline scientifique centrée sur le développement, l’analyse et l’implémentation de méthodes automatisables, offrant la possibilité à une machine d’évoluer grâce à un processus d’apprentissage à partir des données et à effectuer des tâches de façon performante.
	 \item Un traitement statistique de masses de données réunissant à la fois mathématiques appliquées et informatique.
	 \item Utilisé lorsque le Big Data rend inopérant les méthodes statistiques traditionnelles.
	\end{itemize}
\end{block}
Le Machine Learning est composé de plusieurs types d'algorithmes d’apprentissage (Supervisé, non supervisé, semi-supervisé, par renforcement).
	\end{frame}
	
	\section{Graphe de flux et Graph Mining}
	\subsection{Graphe de flux}
	\begin{frame}
		DCbrain emploie une approche basée sur une représentation des réseaux physiques en graphes de flux.\\~\\
		\pause
		Cette représentation en graphe permet de prendre en compte les spécifités des flux du réseau, c'est-à-dire :
		\begin{itemize}
		\pause
		\item retranscrire les données du réseau sous forme de flux
		\pause
		\item analyser (calculer) et représenter les données liées au flux du réseau
		\end{itemize}
		~\\
		\pause
		La représentation du réseau en graphe de flux a pour avantage de : 
		\begin{itemize}
		\pause
		\item repérer beaucoup plus facilement des anomalies dans le réseau
		\pause
		\item simuler des évolutions du réseau
		\end{itemize}
	\end{frame}
	
	\begin{frame}
		Ce genre de graphe peut être utilisé pour tout réseau physique de fluide, par exemple les réseaux électriques :
		\pause
		\begin{exampleblock}{Graphe de flux : d'un réseau électrique :}
		\begin{itemize}
		\item Nœuds : des connections
		\item Arcs : canaux pour acheminer l'électricité (câbles)
		\end{itemize}
		\end{exampleblock}
		~\\
		\pause
		L'analyse des données à partir d'un graphe de flux est réalisée grâce à la méthode de \textbf{graph mining}.
	\end{frame}	
	
	\subsection{Graph Mining}
	\begin{frame}
		Les graphes sont un outil très efficace pour la représentation de structures complexes comme les réseaux physiques dans notre cas.\\~\\
		\pause
		Le graph mining est une méthode d'analyse de données représentées par un graphe.\\ 
		\pause
		Il s'agit d'extraire des sous-graphes qui décrivent l'information recherchée du graphe. Ces informations peuvent ensuite être utilisées pour caractériser et classifier le graphe et effectuer des regroupement, des analyses de fréquence et des recherches de similarités de graphes (dans une base de données de graphes par exemple)\footnote{\href{https://www.datamining.informatik.uni-mainz.de/graph-mining/}{https://www.datamining.informatik.uni-mainz.de/graph-mining/}}
		\footnote{\href{http://web.engr.illinois.edu/\~hanj/cs512/bk2chaps/chapter\_9.pdf}{http://web.engr.illinois.edu/\~hanj/cs512/bk2chaps/chapter\_9.pdf}}
		.
	\end{frame}
	\begin{frame}
		Cette méthode est utilisée dans plusieurs domaines comme les données web (graphes de réseaux sociaux), la chimie (réseaux bilogiques), etc.
		\\~\\
		\pause
		Le graph mining est une forme d'analyse de donnée structurée (\textit{structured data mining} dont le processus consiste à trouver et extraire des informations utiles à partir d'une masse de \textbf{données semi-structurées})\footnote{\href{https://en.wikipedia.org/wiki/Structure\_mining}{https://en.wikipedia.org/wiki/Structure\_mining}}.\\~\\
		\pause
		Une donnée semi-structurée est une forme de données structurées qui ne correspond pas à la structure formelle... \footnote{\href{https://en.wikipedia.org/wiki/Semi-structured\_data}{https://en.wikipedia.org/wiki/Semi-structured\_data}}
	\end{frame}
	
	\section{Conclusion}
	\begin{frame}
		Parler de l'application : on prend un .csv, on fait...
	\end{frame}

\end{document}