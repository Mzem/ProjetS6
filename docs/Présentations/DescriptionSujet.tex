\documentclass{beamer}
\usetheme{Warsaw}
\usepackage[utf8]{inputenc}
\usepackage[T1]{fontenc}
\usepackage[french]{babel}
\usepackage[ddmmyyyy]{datetime}
%\usepackage{beamerthemebars}

%Plan/Sommaire automatique avant chaque section
\AtBeginSection[]{
  \begin{frame}
  \frametitle{Plan}
  \tableofcontents[currentsection]
  \end{frame}
}

\author{Sonny Klotz - Jean-Didier Pailleux - Malek Zemni}
\institute{UVSQ}
\date{\today}
\usepackage{../tex/myInfolines}
\title{Documentation sur le sujet}

\begin{document}

	\begin{frame}
		\titlepage
	\end{frame}
	
	\begin{frame}
		\frametitle{Plan général}
		\tableofcontents
	\end{frame}
	
	\section{Introduction}
	\begin{frame}
		DCbrain développe des outils qui permettent de visualiser ce qui ce passe sur les \textbf{réseaux physiques} afin de :
		\begin{itemize}
		\item trouver et prédire les problèmes sur ces réseaux
		\item optimiser ces réseaux
		\end{itemize}
		~\\
		\pause
		Les données sont collectées à partir des réseaux physiques puis analysées grâce aux technologies du \textbf{Big Data}.\\~\\
		\pause
		\begin{block}{Réseaux physiques}
		Réseaux industriels, de fluide, de distribution. Exemple : distribution pétrolière, gazière, électrique...
		\end{block}
	\end{frame}
	
	\begin{frame} 
		\begin{block}{Big Data}
		Ensembles de très gros volumes de données, à la fois structurées, semi-structurées ou non structurées, qui peuvent être traitées et exploitées dans le but d’en tirer des informations intelligibles et pertinentes.
		\end{block}
		\pause
		~\\
		Exemple de sources : 
		\begin{itemize}
			\pause\item Capteurs pour collecter les informations climatiques, de trafic, consommation (Smart cities, Internet des Objets)
			\pause\item Messages sur les réseaux sociaux 
			\pause\item Enregistrements transactionnels d’achat en ligne 
			\pause\item Signaux GPS de téléphones mobile
		\end{itemize}
	\end{frame}
	
	\begin{frame}
	\frametitle{Problème}
		Comment utiliser et donner du sens à ces masses de données enregistrées sur ces réseaux ?
	\end{frame}
	
	\section{Analyse descriptive de données}
	\subsection{Généralités}
	\begin{frame}
		\begin{block}{Définition}
		Ensemble de techniques de statistique descriptive.
		\end{block}
		~\\
		\begin{itemize}
		\pause\item \textbf{Objectifs} : une description succincte, regrouper les données.
		\pause\item \textbf{Les données} : tableaux de données quantitatives et qualitatives.
		\pause\item \textbf{Avantages} : traitement en masse, représentations graphiques.
		\end{itemize}
	\end{frame}
	
	\subsection{Statistique unidimensionnelle}
	\begin{frame}
		Travail à réaliser :
		\begin{itemize}
		\item détection du type de colonnes 
		\item statistique descriptive sur une colonne
		\end{itemize}
		
		~\\
		\pause
		
		données quantitatives -> effectifs cumulés et fréquence d'apparition
		
		~\\
		\pause
		
		données quantitatives
		\begin{itemize}
		\item tendance centrale : moyennes et mode
		\item dispersion : quantiles, variance et écart-type
		\item forme : symétrie et aplatissement
		\end{itemize}
		
		~\\
		\pause
		
		\textbf{Open source} : R - Python - ROOT (C++ et Python) - Java
		
	\end{frame}
	
	\section{Big Data et Machine Learning}
	\subsection{Big Data}
	\begin{frame}
		\begin{block}{Big Data}
		Le Big Data fait référence à la masse de données collectée. On considère du Big Data quand le traitement devient trop long pour une seule machine.
		\end{block}
		~\\
		\pause
		Les traitements de cette quantité importante de données est massivement "parallélisé" avec MapReduce/Hadoop.
		\\~\\
		\pause
		Le Big Data est caractérisé par les 3V :
		\begin{itemize}
			\pause\item le Volume de données considérable à traiter.
			\pause\item la Variété de ces données qui peuvent être brutes, non structurées ou semi-structurées
			\pause\item la Vélocité qui désigne le fait que ces données sont produites, récoltées et analysées en temps réel.
		\end{itemize} 
	\end{frame}
	\subsection{Machine Learning}
	\begin{frame}
		\begin{block}{Le Machine Learning est :}
			\begin{itemize}	
				\pause \item Une discipline scientifique centrée sur le développement, l’analyse et l’implémentation de méthodes automatisables, offrant la possibilité à une machine d’évoluer grâce à un processus d’apprentissage à partir des données et à effectuer des tâches de façon performante.
				\pause\item Un traitement statistique de masses de données réunissant à la fois mathématiques appliquées et informatique.
				\pause\item Utilisé lorsque le Big Data rend inopérant les méthodes statistiques traditionnelles.
			\end{itemize}
		\end{block}
		~\\
		\pause
		Le Machine Learning est composé de plusieurs types d'algorithmes d’apprentissage (supervisé, non supervisé, semi-supervisé, par renforcement).
	\end{frame}
	
	\section{Graphe de flux et Graph Mining}
	\subsection{Graphe de flux}
	\begin{frame}
		DCbrain emploie une approche basée sur une représentation des réseaux physiques en \textbf{graphes de flux}.
		\\~\\
		\pause
		Les graphes de flux permettent de représenter les données liées au flux du réseau et de les analyser.
		\\~\\
		\pause
		Avantages :
		\begin{itemize}
		\pause
		\item repérer beaucoup plus facilement des anomalies dans le réseau
		\pause
		\item simuler des évolutions du réseau
		\end{itemize}
	\end{frame}
	\begin{frame}
		Les graphes de flux peuvent être utilisés pour tout réseau physique de fluide, par exemple les réseaux électriques :
		\pause
		\begin{exampleblock}{Graphe de flux : d'un réseau électrique :}
		\begin{itemize}
		\item Nœuds : des connections
		\item Arcs : canaux pour acheminer l'électricité (câbles)
		\end{itemize}
		\end{exampleblock}
	\end{frame}	
	
	\subsection{Graph Mining}
	\begin{frame}
		Les méthodes d'analyse de données classiques sont limitées au \textbf{données structurées}.
		\\~\\
		\pause
		Les données représentées par des graphes sont dites \textbf{semi-structurées}.
		\\~\\
		\pause
		Il est nécessaire d'employer une méthode d'analyse qui convient à ce genre de données : le \textbf{graph mining}.
		\\~\\
		\pause
		Principe du graph mining :
		\begin{itemize} 
		\pause \item extraire des informations utiles à partir d'une masse de données semi-structurées
		\pause \item càd miner des sous-graphes fréquents qui décrivent l'information recherchée
		\end{itemize}
	\end{frame}
	\begin{frame}
		Utilisation des informations extraites :
		\begin{itemize}
			\pause \item Trouver des relations entre différents éléments du graphe
			\pause \item Prédire le comportement des éléments d'un graphe
		\end{itemize}
		~\\
		\pause
		Domaines d'application :
		\begin{itemize}
			\pause \item Réseaux sociaux
			\pause \item Chimie et biologie
			\pause \item Réseaux physiques
		\end{itemize}
		~\\
		\pause
		Quelques algorithmes :
		\begin{itemize}
			\pause \item Apriori-based Approach
			\pause \item Pattern-Growth Approach
		\end{itemize}
	\end{frame}
	
	\section{Conclusion}
	\begin{frame}
		Et notre application...
	\end{frame}

\end{document}
