\documentclass{beamer}
\usetheme{Warsaw}
\usepackage[utf8]{inputenc}
\usepackage[T1]{fontenc}
\usepackage[french]{babel}
\usepackage[ddmmyyyy]{datetime}
%\usepackage{beamerthemebars}

%Plan/Sommaire automatique avant chaque section
\AtBeginSection[]{
  \begin{frame}
  \frametitle{Plan}
  \tableofcontents[currentsection]
  \end{frame}
}

\author{Sonny Klotz - Jean-Didier Pailleux - Malek Zemni}
\institute{UVSQ}
\date{\today}
\usepackage{../tex/myInfolines}
\usepackage{longtable,array}
\title{Présentation Cahier des Charges}

\begin{document}

	\begin{frame}
		\titlepage
	\end{frame}
	
	\section{Introduction}
		\begin{frame}
			JD (2:00)
		\end{frame}
	
	\section{Architecture}
		\begin{frame}
			JD (4:30 - 5:00)
		\end{frame}	
	
	\section{Outils, langages de programmation} 
		\begin{frame} 
			\textbf{Contraintes}
			\begin{enumerate}
				\item Le produit doit fournir une application web.
				\item Le produit doit être développé avec un langage de programmation compatible avec l’analyse de données.
				\item Le produit doit fournir des API pour le chargement et l’analyse de données.
			\end{enumerate}
			~\\
			\textbf{Outils et langages de programmation}
			\begin{itemize}
				\item \textbf{Python :} adapté pour l’ADD et le développement d’applications web
				\item \textbf{Flask :} framework web Python
				\item \textbf{HTML et CSS :} présentation et mise en forme des pages web
				\item \textbf{JavaScript :} dynamiser les pages web
				\item \textbf{jQuery :} gestion des événements et \textbf{Ajax}
				\item \textbf{c3js :} module de représentations graphiques
				\item \textbf{Sphinx :} framework Python, génération de documentation
			\end{itemize}
		\end{frame}
	
	\section{Fonctionnement de l'application}
		\begin{frame}
			Malek (3:30)
		\end{frame}
	
	
	\section{Bilan technique}
		\begin{frame}
			Sonny (4:30)\\
			(3:00) : points délicats\\
			(2:00) : problèmes\\
		\end{frame}
	
	\section{Organisation interne du groupe}
		\begin{frame}
			Sonny (00:30)
		\end{frame}
	
	\section{Coûts}
		\begin{frame}
			Sonny (1:00)
		\end{frame}
	
	\section{Conclusion}
		\begin{frame}
			Sonny (1:00)
		\end{frame}
	
\end{document}
