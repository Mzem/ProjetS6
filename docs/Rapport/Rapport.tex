\input{../tex/preambule}

\title{\vspace{\fill}\textbf{\Huge Compte Rendu}}
\author{
	Sonny Klotz - Jean-Didier Pailleux - Malek Zemni
	\vspace{2em}\\
	\textit{Interface de chargement, de contrôle}\\\textit{et d’analyse statistique des données}\\\textit{pour la constitution d’un graphe de flux}
	\vspace{2em}
}

\begin{document}
\pagenumbering{gobble}\clearpage
\maketitle\vspace{9em}
\begin{center}\includegraphics[scale=0.7]{../Cahier/logo.png}\end{center}
\begin{flushright}Module \textit{Projet}\end{flushright}
\newpage
\tableofcontents
\newpage\clearpage\pagenumbering{arabic}

	\section*{Introduction}
	% pour l'intro si tu modifies des trucs, garde la meme mise en page malek.
	\paragraph{}Ce document est le compte-rendu final de notre travail qui s'inscrit dans le cadre du module \textit{Projet} de la licence informatique de l'\textit{\textbf{UVSQ}}. Le sujet de ce projet a été proposé par l'entreprise \textit{\textbf{DCbrain}}.
	\paragraph{}Ce projet découle d'un thème d’actualité, le \textbf{Big Data}. La notion Big Data fait référence aux masses de données collectées et l'enjeu consiste à extraire une valeur ajoutée de ces données.

	Les techniques utilisées pour répondre au problème sont empruntées aux statistiques et l'on va parler d'\textit{\textbf{analyse descriptive de données}}\footnote{On utilisera l'acronyme ADD par la suite pour parle d'analyse descriptive de données.}: une branche des statistiques qui étudie les données par le biais d'informations synthétiques, sans supposer qu'elles suivent un comportement particulier.
	\paragraph{}Notre client, \textit{\textbf{DCbrain}}, emploie une approche basée sur une représentation des \textbf{réseaux physiques} de leurs clients en \textbf{graphes de flux}. Les graphes de flux permettent de représenter les données liées au flux du réseau, repérer les anomalies sur le réseau et simuler les évolutions de celui-ci.
	
	Les graphes de flux peuvent être utilisés pour tout réseau physique de fluide, tels les réseaux électriques et c’est pourquoi les clients de \textit{\textbf{DCbrain}} sont des groupes comme ERDF ou la SNCF. Par exemple, on pourra représenter les canaux qui acheminent l'électricité par des arcs et les connexions par des nœuds.
	\paragraph{}L'objectif de ce projet est de fournir aux utilisateurs une application web. Ce produit représente un outil complémentaire au travail de \textit{\textbf{DCbrain}}, dont les fonctionnalités permettent aux utilisateurs de charger des données, les visualiser et les analyser. 

	\paragraph{}Dans la première partie de ce document, on présentera l'architecture de notre application, illustrée par un organigramme qui a été préalablement établi. Cet organigramme représente un découpage de l'application en modules avec les différentes informations qui circulent entre ces modules.
	
	Ensuite, on parlera des différents langages de programmation choisis pour réaliser l'application et des contraintes qui ont justifié ces choix.
	
	La troisième partie du compte-rendu évoquera les points délicats du projet, autrement dit les concepts informatiques complexes utilisés.
	Le fonctionnement de l'application sera décrit dans la quatrième partie.
	Après cela, on fera le bilan technique de notre projet, c'est à dire, l'application, son fonctionnement et ses problèmes.
	
	Finalement, dans les deux dernières parties, on établira un bilan quant à l'organisation interne au sein du groupe et un bilan comparatif des coûts présumés et des coûts finaux de notre produit.
	
	\paragraph{}En addition au code source, un manuel utilisateur et une documentation sont livrés. La documentation se présente sous deux formes, \textit{\textbf{pdf}} et \textit{\textbf{html}}. La version \textit{\textbf{html}} est plus lisible grâce à l'utilisation de thèmes disponibles avec l'outil qui nous a permis de générer ce document.
	
	\section{Architecture de l'application}
		\subsection{Organigramme et données échangées}
		Cet organigramme a été préalablement établi dans le cahier des charges du produit. Il représente la décomposition en modules de l'application, ainsi que les informations qui circulent entre ces modules.
		
		On notera des arrangements faits dans la représentation des informations transmises entre les modules de l'organigramme. Ces arrangements sont principalement effectués pour des fins de compatibilité avec les outils utilisés et seront explicitement justifiés.
		\begin{figure}[H]
			\begin{tikzpicture}
			\begin{scope}[xscale=2,yscale=1.5]	
				\node (IW) at (0,5) [rectangle,draw,text depth=3cm,minimum width=16cm,minimum height=4cm,font=\textbf\Large] {\begin{tabular}{c}Interface web\end{tabular}};
				\node (F2) [rectangle,draw,dashed] at ([yshift=0.4cm]IW.center) {\begin{tabular}{c}Fenêtre rôle et choix colonne \end{tabular}};
				\node (F1) [rectangle,draw,dashed] at ([xshift=-4cm]F2.east) {\begin{tabular}{c}Fenêtre choix fichier\end{tabular}};
				\node (F3) [rectangle,draw,dashed] at ([xshift=4.1cm]F2.west) {\begin{tabular}{c}Fenêtre résultats ADD\end{tabular}};
				\node (MAIN) [rectangle,draw,dashed,below=of F2.south,yshift=0cm] {\begin{tabular}{c}Gestion des flux\end{tabular}};

			
				\node (API1) at (-2.5,0) [rectangle,draw,fill=blue!25,text depth=-3.5cm,minimum width=5cm,minimum height=5cm,font=\textbf\Large] {\begin{tabular}{c}API 1\\Chargement des données\end{tabular}};
				\node (VERIF) [rectangle,draw,dashed,fill=white] at ([yshift=1cm]API1.center){\begin{tabular}{c}Vérification format\\fichier\end{tabular}};
				\node (ANALYS) [rectangle,draw,dashed,fill=white,below=of VERIF.south,yshift=0.5cm] {\begin{tabular}{c}Analyse contenu\\fichier\end{tabular}};
			
				\node (API2) at (2.5,0) [rectangle,draw,fill=blue!25,text depth=-5cm,minimum width=5cm,minimum height=6.5cm,font=\textbf\Large] {\begin{tabular}{c}API 2\\Analyse descriptive des données\end{tabular}};
				\node (ADD1) [rectangle,draw,dashed,fill=white] at ([yshift=1.7cm]API2.center){\begin{tabular}{c}ADD qualitatives\end{tabular}};
				\node (ADD2) [rectangle,draw,dashed,fill=white,below=of ADD1.south,yshift=0.5cm] {\begin{tabular}{c}ADD quantitatives\\discrètes\end{tabular}};
				\node (ADD3) [rectangle,draw,dashed,fill=white,below=of ADD2.south,yshift=0.5cm] {\begin{tabular}{c}ADD quantitatives\\continues\end{tabular}};
			
				\draw[-triangle 45] (F1.south east) -- node[anchor=south,left]{9} (MAIN.north west);
				\path[->,>=stealth'] (MAIN) edge[bend left=10] node[anchor=south,left]{10} (F2);
				\path[->,>=stealth'] (F2) edge[bend left=10] node[anchor=south,right]{11} (MAIN);
				\path[->,>=stealth'] (MAIN) edge[bend left=10] node[anchor=south,right]{12} (F3);
				\path[->,>=stealth'] (F3) edge[bend left=10] node[anchor=south,right]{13} (MAIN);
			
				\draw[-triangle 45,red] (MAIN.south west) -- node[anchor=south,left] {1} (API1.90);
				\draw[-triangle 45,red] (API1) -| node[anchor=south,below] {2} (MAIN.215);			
			
				\draw[-triangle 45] (MAIN.south east) -- node[anchor=south,right] {3} (ADD1.80);
				\draw[-triangle 45] (ADD1.110) -- node[anchor=south,left] {4} (MAIN.343);
				\draw[-triangle 45] (MAIN.335) |- node[anchor=south,right] {5} (ADD2.175);
				\draw[-triangle 45] (ADD2.185) -| node[anchor=south,below] {6} (MAIN.330);				
				\draw[-triangle 45] (MAIN.290) |- node[anchor=north,right] {7} (ADD3.175);
				\draw[-triangle 45] (ADD3.185) -| node[anchor=south,below] {8} (MAIN.270);
			
			\end{scope}
			%Légende
			\begin{scope}
				\node (LEGENDE) at (-7,-5) {\textbf{Légende :}};
				\node (FAMILLE) at (-4.5,-5) [rectangle,draw] {\begin{tabular}{c}Package\end{tabular}};
				\node (MODULE) at (-2,-5) [rectangle,draw,dashed] {\begin{tabular}{c}Module\end{tabular}};
				\path[->,>=stealth'] (0.5,-5.3) edge[bend left=0] node[anchor=south,above]{informations transmises} (3,-5.3);
				\path[->,>=stealth',red] (0.5,-6.3) edge[bend left=0] node[anchor=south,above]{correction} (3,-6.3);
			\end{scope}
			\end{tikzpicture}
			\caption{Organigramme des différents modules du logiciel}\label{fig:M1}
		\end{figure}
			
		\textbf{Notes :}\\
			\textbf{({\color{red}1})} Chemin du fichier CSV importé : lancement des fonctionnalités du module de chargement de manière indépendante : la vérification de son format et l'analyse de son contenu\\
			\textbf{({\color{red}2})} Le résultat du chargement du fichier CSV importé :
			\begin{itemize}
				\item soit un message d'erreur signalant un problème lié au fichier
				\item soit deux structures, l'une contenant les données du fichier, et l'autre contenant une description du nom, du type, et des erreurs de chaque colonne de données du fichier
			\end{itemize}
			{\color{red}Justification de la modification :} les outils utilisés pour faire communiquer les modules ne permettaient pas la transmission d'un certain type d'objets, dont les objets représentant les fichiers. On a donc décidé de récupérer directement les résultats de l'analyse du fichier et d'y inclure la gestion des potentielles erreurs.\\
			\textbf{(3)} Ensemble de données de type qualitatif\\
			\textbf{(4)} Erreur ou effectifs, effectifs cumulés, fréquences, fréquences cumulées\\
			\textbf{(5)} Ensemble de données de type quantitatif discret\\
			\textbf{(6)} Erreur ou indicateurs de tendance central, de dispersion et de forme, les anomalies, la distribution des données, un diagramme à moustaches\\
			\textbf{(7)} Ensemble de données de type quantitatif continu\\
			\textbf{(8)} Même données que (6)\\
			\textbf{(9)} Chemin du fichier CSV importé \\
			\textbf{(10)} Informations du (2) et ensemble de données contenu dans le fichier CSV \\
			\textbf{(11)} Signal de validation du choix de la colonne, et noms des colonnes\\
			\textbf{(12)} Envoi des résultats d'analyses de (4), (6) et (8)\\
			\textbf{(13)} Signal de demande d'exportation des résultats de l'ADD ou analyse d'une autre colonne\\		
		
		\subsection{Fonctionnalités des modules}
			
			\subsubsection*{Package Chargement des données}
			\begin{enumerate}[leftmargin=*]
				\item Module Vérification format fichier :\\
				Vérification de l'existence du fichier, de l'extension, de la lisibilité du contenu (texte brut ou formaté) et l'accessibilité en lecture.
				\item Module Analyse contenu fichier :\\
				Ce module comprend deux fonctionnalités principales :
				\begin{itemize}[leftmargin=0.2cm]
					\item Lecture du contenu du fichier CSV : on détermine le délimiteur de données du fichier, puis on lit ses données ligne par ligne. Ces données sont stockées dans une \textbf{première structure}.
					\item Description des données de chaque colonne du fichier CSV : on stocke les noms des colonnes fournis dans le fichier, ensuite on détermine les types de données attendus à partir de ces noms, et finalement, on parcourt la structure contenant les données du fichier en comparant le type de ces données au type attendu. Si ces types ne correspondent pas, on indique une description de l'erreur. Ces trois informations (nom, type et erreurs) sont stockées dans une \textbf{deuxième structure}.
				\end{itemize}
				Ce module va donc fournir les deux structures décrites ci-dessus.
			\end{enumerate}
			
			\subsubsection*{Package Analyse descriptive de données}
			\begin{enumerate}[leftmargin=*]
				\item Module ADD qualitatives :
					\begin{itemize}[leftmargin=0.2cm]
					\item Calcul des effectifs, effectifs cumulés, fréquences et fréquences cumulées des données
					\item Préparation des représentations graphiques : diagramme en secteur et histogramme
					\end{itemize}
				\item Module ADD quantitatives discrètes :
					\begin{itemize}[leftmargin=0.2cm]
					\item Statistiques : moyenne, quantiles, variance, écart-type, coefficients de symétrie et d'aplatissement de Fisher
					\item Anomalies de Tukey : toute donnée hors de l'intervalle [Q1 - 1.5*IQ ; Q3 + 1.5*IQ], où Q1 et Q3 sont les quartiles, et IQ l'écart inter-quartiles.
					\item Préparation des représentations graphiques : fonction de distribution, fonction de distribution cumulative, boîte à moustaches
					\end{itemize}
				\item Module ADD quantitatives continues :
					\begin{itemize}[leftmargin=0.2cm]
					\item Discrétisation de l'étendue (découpage en classe d'intervalles)
					\item Calcul des quantiles pour le cas  de données continues
					\item Préparation des représentations graphiques : fonction de distribution cumulative, boîte à moustaches
					\end{itemize}
			\end{enumerate}
			
			\subsubsection*{Package Interface web}
			\begin{enumerate}[leftmargin=*]
				\item Module Gestion des flux :
					\begin{itemize}[leftmargin=0.2cm]
					\item Gestion des branchements : exécution normale ou arrêts pour cause d'erreur.
					\item Interface entres les différentes fonctionnalités : communique les données nécessaires entre les modules.
					\end{itemize}
				\item Module Fenêtre choix fichier :
					\begin{itemize}[leftmargin=0.2cm]
					\item Récupération un fichier CSV en renseignant son chemin en parcourant l'arborescence de fichiers, ou de la manière d'un Drag \& Drop.
					\end{itemize}
				\item Module Fenêtre rôle et choix colonne :
					\begin{itemize}[leftmargin=0.2cm]
					\item Affichage des données du fichier CSV : noms de colonnes, leurs valeurs et le nombres de lignes du fichier.
					\item Affichage des données erronés (description de l'erreur).
					\item Affichage d'un échantillon de données du fichier à l'aide de filtres.
					\item Sélection et envoi d'une colonne de mesures pour lancer l'analyse sur celle-ci. 
					\end{itemize}
				\item Module Fenêtre résultats ADD :
					\begin{itemize}[leftmargin=0.2cm]
					\item Affichage des résultats d'analyse descriptive : informations statistiques de l'API 2 et représentations graphiques pour visualiser les données dans leur ensemble.
					\item Une fonctionnalité de retour en arrière permet de sélectionner une nouvelle colonne sans relancer l'applet.
					\item Téléchargement des analyses descriptives effectuées au format \lstinline!.csv!
					\end{itemize}
				\end{enumerate}
		
	\section{Outils et langages de programmation}
		\subsection{Contraintes}
		Lors de la réalisation de l'application, des contraintes ont dû être respectées. Ces contraintes ont été imposées par notre client \textbf{\textit{DCbrain}} lors de la remise du sujet du projet :
		\begin{enumerate}[leftmargin=*]
			\item Le produit doit fournir une application web : notre produit sera une application fonctionnant sur un navigateur web, appelée \textbf{\textit{applet}}.
			\item Le produit doit être développé avec un langage de programmation compatible avec l'analyse de données : le langage de programmation choisi doit inclure des bibliothèques qui permettent d'analyser les données (taches de data mining et de machine Learning).
			\item Le produit doit fournir des API (ensemble de packages) pour le chargement et l'analyse de données : le code source de l'application doit intégrer des API d'analyse descriptive de données qui pourront être réutilisés par le client.
		\end{enumerate}	
		Ces contraintes se sont avérées déterminantes pour les choix techniques que l'on a fait, et en l'occurrence, le choix des langages de programmation. 
		
		\subsection{Choix des outils et des langages}
		Les contraintes imposées par le client ainsi que les exigences définies nous ont permis de fixer nos choix sur plusieurs langages de programmation qui vont interagir ensemble : 
		\begin{itemize}[leftmargin=*]
			\item \textbf{Python :} d'une part, ce langage est adapté pour l'ADD, puisqu'il est doté de nombreux modules permettant d'effectuer du calcul scientifique. D'autre part, \textbf{\textit{Python}} est compatible avec le développement d’applications web avec l'aide de frameworks dédiés.
			\item \textbf{Flask :} framework web qui s’efforce d’être la plus simple possible, à l'opposé d'autres bibliothèques plus complètes. C'est donc plus abordable pour une première expérience sur le développement d'applications web.
			\item \textbf{HTML et CSS :} l'application web nécessite le recours au langage de balisage \textbf{\textit{HTML}} pour la présentation et à \textbf{\textit{CSS}} pour la mise en forme des pages web.
			\item \textbf{JavaScript :} c'est le langage qui nous permet de dynamiser nos pages web rendues côté client, c'est pour cela qu'on parle d'application web.
			\item \textbf{jQuery :} librairie \textbf{JavaScript} pour une manipulation simplifiée de concepts avancés, dont la gestion des évènements et \textbf{\textit{Ajax}} (concept détaillé dans la prochaine partie).
			\item \textbf{c3js :} module de représentations graphiques \textbf{JavaScript} réutilisable. Il est basé sur \textbf{\textit{d3js}}, une librairie \textbf{JavaScript} pour une manipulation de documents dirigée par les données.
			\item \textbf{Sphinx :} framework \textbf{Python} pour la génération de documentation. Sa puissance vient de la génération automatique de la documentation à partir du code source.
		\end{itemize}
		
			
	\section{Fonctionnement de l'application}

		\subsection{Format du fichier CSV}
			Notre application traite des données contenues dans un fichier CSV. Le format du fichier a été établi par le client \textbf{\textit{DCbrain}}. Son contenu est structuré et est décrit par des colonnes aux types prédéfinis :
			\begin{center}\vspace{-1.2em}\footnotesize\begin{longtable}{|>{\centering}m{5cm}|>{\centering}m{2cm}|>{\centering}m{2cm}|>{\centering}m{2.5cm}|>{\centering\arraybackslash}m{2cm}|}			
				\hline \multicolumn{1}{|c|}{\textbf{timestamp}} & \multicolumn{1}{c|}{\textbf{parent}} & \multicolumn{1}{ c|}{\textbf{enfant}} & \multicolumn{1}{c|}{\textbf{mesure 1}} & {\textbf{mesure 2}} \\
				\hline 	January 1st 2017, 15:00:00.000 & 102 & 95 & 26644.235 & 176.253\\
				\hline
			\end{longtable}\vspace{-3.5em}\end{center}
			\paragraph{} Le graphe de flux utilisé par \textbf{\textit{DCbrain}} pour analyser les réseaux de ses clients est orienté, d'où l'utilisation des nœuds \textit{parent - enfant} numérotés. Ceux-ci permettent d'identifier une connexion précise. Les colonnes \textit{mesure} représentent des données mesurées sur une connexion à un temps donné. Le nombre de ces colonne n'est pas limité et en pratique il ne dépassera pas la dizaine.

		\subsection{Interface web}
			Notre application est destinée à des utilisateurs n'ayant pas de connaissances pré-requises pour sa manipulation. Des capacités à interpréter des informations statistiques basiques et à lire de représentations graphiques sont cependant demandées. Son utilité va être déterminée par des graphiques et des présentations de valeurs intuitifs qui vont guider l'utilisateur dans l'interprétation des résultats.
			\paragraph{}Le scénario standard d'utilisation est simple. Une première fenêtre laisse le choix à l'utilisateur de sélectionner un fichier au format défini à la sous-partie précédente.
			
			Si le fichier n'est pas au bon format, il ne sera pas traité et il faudra sélectionner un autre fichier. Sinon, la fenêtre suivante peut être affichée. Sur cette fenêtre sont affichées dans un tableau les données du fichier qui peuvent être soumis à différents traitements : renommage de l'en-tête des colonnes et filtrage des valeurs.
			
			Ensuite, on peut valider l'analyse sur les valeurs filtrées pour passer à la dernière fenêtre. Cette fenêtre dispose d'un système de navigation entre les différents graphiques disponibles, et un compte-rendu synthétique des analyses descriptives. 
			
			Finalement, on a le choix de revenir à la fenêtre de filtrage des valeurs, ou bien de télécharger des résultats complets d'ADD sur notre ordinateur.
			
		\subsection{API de chargement des données et d'analyses descriptives}
			Une partie de notre travail consiste à fournir des paquets réutilisables pour le chargement des données et l'ADD. Pour mettre en avant nos résultats, nous avons décidé d'écrire un script en réutilisant les fonctionnalités de nos API.
			
			Son principe est d'effectuer des analyses descriptives sur chaque colonne pour chaque arc distinct d'un fichier toujours au format présenté plus tôt. Ce script écrit dans un nouveau fichier CSV en sortie, dans un format similaire pour permettre une réutilisation et une interprétation facile.
			
			Les résultats satisfaisants de ce script nous ont incités à le réutiliser pour la fonctionnalité de téléchargement des analyses de notre application.
		
		
	\section{Bilan technique du projet}
		Notre produit final, c'est à dire l'application, se comporte comme prévu : l'application est fonctionnelle, la liaison entre ses différents modules réussit bien et les différentes fonctionnalités fournissent le résultat attendu.
		
		Nous aurions souhaité enrichir les résultats d'analyses descriptives avec des graphes originaux. Cependant, leur conception nécessite une connaissance avancée d'outils graphiques pour le web, ce que nous avons pas eu le temps d'approfondir.
		\paragraph{Tests unitaires:} les modules de notre application ont été testé et le code des tests est disponible avec notre code source. Dans la partie suivante, on va détailler certains problèmes qui ont été résolus grâce à ces tests.
		\subsection{Points délicats:}
	La majorité des outils et des langages de programmation utilisés étaient nouveaux pour nous, ce qui a nécessité un temps d'apprentissage et d'adaptation. Cependant, on a finalement réussi à se familiariser avec ces outils pour assurer la réalisation de notre application.\\
	
	Dans le cadre de cette application web, le paradigme client serveur était un point assez délicat à traiter. Le client est considéré ici comme le navigateur, et envoie des requêtes au serveur qui héberge le site. Le serveur répond en renvoyant la page web demandée. 
	\paragraph{}Ce sont des applications côté client exécutées par le navigateur. HTML met en place des balises spéciales, \lstinline!<script>! dans lequel du code va être exécuter une fois la page chargée.
	Pour résumer nous avons d'un côté du code côté serveur qui permet de rendre des templates formatés, et d'autre part du code sur le navigateur qui peut s'exécuter pour obtenir des pages dynamiques. Cela dit, ce sont deux parties indépendantes, pour obtenir une communication de données entre ces deux entités, sans avoir à faire une requête de page web entière, il faut comprendre \textbf{Ajax}.\\
	\paragraph{}Pour obtenir \textbf{Ajax} (Asynchronous JavaScript And XML).Ce sont des requêtes qui envoi uniquement des données au format XML nécessaire (pas toute une page web), donc raffraichissement partiel de la page. Code plus léger et moins de chargement.
		

		\subsection{Problèmes rencontrés}
			\subsubsection*{Problèmes résolus :} 
			Lors de la réalisation de l'application, on a été confrontés à plusieurs problèmes et points délicats, principalement des verrous techniques, qui ont perturbé le bon déroulement de notre travail :
			\begin{itemize}[leftmargin=*]
				\item Les premiers verrous rencontrés sont liés aux concepts informatiques. Ces concepts n'étaient pas maîtrisés voire inconnus avant d'aborder le projet, ils ont donc grandement faussé l'estimation quant à la charge de travail que représente le projet. 
				\item Lors de la manipulation de ces outils, on a rencontré des problèmes de compatibilité lors de l'interaction entre les différents langages de programmation utilisés. Un des problèmes principaux était le type de données transférées entre les différents modules.\\
				Par exemple, la bibliothèque \textit{\textbf{Flask}} ne permet pas la transmission d'objets qu'elle considérait comme non sérialisable (en particulier les fichiers) et ceci a demandé des aménagements dans le schéma de transfert d'informations. Les fonctions manipulant des fichiers ont dû être modifiées pour utiliser d'autres structures compatibles (d'où une divergence avec les spécifications).
				\item Les tests ont pu mettre révéler un défaut de précision des nombres flottants. Les fréquences cumulées étaient calculées en additionnant plusieurs fois les fréquences d'apparition, ce qui propage les imprécisions de calcul. Ce problème a été résolu en passant par les effectifs cumulés pour n'avoir qu'un calcul flottant à effectuer.
				\item Un dernier problème concerne le graphe des séries temporelles réalisé avec \textit{\textbf{c3js}}. Nous ne pouvons pas avoir de représentation logique si plusieurs valeurs sont mesurées au même instant. Cela dit, sur un même arc la série temporelle est cohérente, et nous avons pu résoudre le problème en mettant à disposition une série temporelle par connexion différente.
			\end{itemize}
				
			\subsubsection*{Problèmes non résolus :}
				Certains problèmes rencontrés n'ont pas été entièrement résolus. Ces problèmes ne sont pas déterminants pour l'acceptabilité de notre produit et sont surtout liés aux performances.
				\paragraph{}Lors du chargement d'un fichier de grand volume, un temps est nécessaire pour que le système d'affichage des données et de filtrage soit opérationnel. L'augmentation du temps d'attente n'est pas linéaire en la taille du jeu de données mais le chargement reste moins performant. Nous aurions aimé atteindre des bonnes performances sur des gros volumes pour avoir une application compétitive.
				\paragraph{}Une représentation graphique aurait été un réel atout pour notre application : les boîtes à moustaches de Tukey.
				Bien qu'il soit annoncé dans les spécifications, il n'est cependant pas disponible dans la librairie \textit{\textbf{c3js}}, et donc n'a pu être réalisé dans les délais.
						
				Nous avons dans notre travail préparé toutes les données nécessaires à cette représentation, y compris les requêtes \textit{\textbf{Ajax}} pour envoyer ces données au navigateur. Cela dit, le code \textit{\textbf{JavaScript}} nécessite la maîtrise d'une librairie telle \textit{\textbf{c3js}} pour faire du graphisme. Malgré la puissance de ce type de librairie, nous n'avons pas pu être assez performant avec pour obtenir des graphismes interactifs originaux dont la qualité et l'esthétisme sont assez satisfaisants. D'où le retrait de la fonctionnalité.
		
				
	\section{Organisation interne du groupe}
	Ce tableau établi dans le cahier des charge, illustre l'assignation des modules pour chaque membre du groupe :
	\begin{center}\vspace{-1em}\footnotesize\begin{longtable}{|>{\centering}m{5cm}|>{\centering}m{2cm}|>{\centering}m{2cm}|>{\centering}m{2.5cm}|>{\centering\arraybackslash}m{1cm}|}			
		\hline \multicolumn{1}{|c|}{\textbf{Module}} & \multicolumn{1}{c|}{\textbf{Malek}} & \multicolumn{1}{ c|}{\textbf{Sonny}} & \multicolumn{1}{c|}{\textbf{Jean-Didier}} & {\textbf{Total}} \\
		\hline 	Gestion des flux & ~ & ~ & x & 1\\
		\hline 	Fenêtre choix fichier & ~ & ~ & x & 1\\
		\hline 	Fenêtre rôle et choix colonne & x & ~ & ~ & 1\\
		\hline 	Fenêtre résultats ADD & ~ & x & ~ & 1\\
		\hline  ADD qualitatives & ~ & ~ & x & 1\\
		\hline 	ADD quantitatives discrètes & ~ & x & ~ & 1\\
		\hline 	ADD quantitative continues &  ~ & x & ~ & 1\\
		\hline 	Vérification format fichier & x & ~ & ~ & 1\\
		\hline 	Analyse contenu fichier & x & ~ & ~ & 1\\
		\hline
	\end{longtable}\vspace{-2.2em}\end{center}
	Cette répartition a été parfaitement respectée. Elle nous a permis de travailler efficacement et assez indépendamment, ce qui prouve que l'assignation des modules a été judicieusement faite. Nous sommes également restés en contact pendant toute la phase de développement pour s'entraider pour la prise en main des nouveaux outils.
	
	\section{Coûts}
	Ce tableau indique les coûts estimés lors de l'établissement du cahier des charges et les coûts finaux, en nombre de lignes de code et pour chaque module :
	\begin{center}\vspace{-1em}\footnotesize\begin{longtable}{|>{\centering}m{3cm}|>{\centering}m{3cm}|>{\centering}m{7cm}|>{\centering\arraybackslash}m{1.5cm}|}			
		\hline \multicolumn{1}{|c|}{\textbf{Module}} & \multicolumn{1}{c|}{\textbf{Nombre de lignes}} & \multicolumn{1}{c|}{\textbf{Justification}} & \multicolumn{1}{c|}{\textbf{Coût final}}\\
		\hline 	Gestion des flux & 15 & Mise en forme du main, appel de l'application, gestion des routes pour flask et des échanges de données avec ajax & \textbf{98}\\
		\hline 	Fenêtre choix fichier & 30 & Fonctions pour : Drag\& Drop + Système de fichiers & \textbf{72}\\
		\hline 	Fenêtre rôle et choix colonne & 65 & Communication avec le module application + Affichage de l'interface + lecteurs des valeurs + affichage des valeurs & \textbf{180}\\
		\hline 	Fenêtre résultats ADD & 100 & Envoi d'informations au module application + Construction des graphes d'ADD & \textbf{200}\\
		\hline  ADD qualitatives  & 60 & Fonctionnalités pur occurrences et fréquences + données des graphes & 53\\
		\hline 	ADD quantitatives discrètes & 100 & Formules attachées à l'ADD quantitatives discrètes + données des graphes & 89\\
		\hline 	ADD quantitatives continues & 85 & Formules attachées à l'ADD quantitatives continues + données des graphes & 93\\
		\hline 	Vérification format fichier & 30 & Ouverture fichier + vérification si ouverture en lecture + présence de texte formaté ou non & 35\\
		\hline 	Analyse contenu fichier & 60 &  Recopie et vérification + initialisation de la structure + Parcours du fichiers avec condition + Fonction pour donner nom et type de colonne & 70\\
		\hline \textbf{Coût Total} & \textbf{545} & \textbf{Estimation totale du coût} & \textbf{850}\\
		\hline 	
	\end{longtable}\vspace{-2.2em}\end{center}
	Les coûts estimés pour les modules effectuant de simples calculs sont respectés (\textit{\textbf{Python}} côté serveur). Par contre, les modules de l'interface web ont plus de lignes que prévu (\textit{\textbf{HTML}} et \textit{\textbf{JavaScript}}). Ceci est dû à l'implémentation de fonctionnalités d'affichage dynamique dans nos fenêtres et les échanges de données \textit{\textbf{Ajax}} qui ont alourdi le code source sans que nous ayons réussi à le prédire.
	
	\section*{Conclusion}
	Ce document résume tout ce qui a été établi avant, pendant et après la réalisation de notre œuvre. Cette œuvre est une application web de visualisation et d'analyse de données. Son développement a été réparti en plusieurs modules indépendants, ce qui facilite sa maintenance et permet ainsi au client d'étendre ses différentes fonctionnalités.
	\paragraph{} Dans le point où l'on est actuellement, on peut revoir, avec du recul, la manière avec laquelle notre produit a été réalisé. En effet, les verrous majoritaires concernent les échanges de données avec \textit{\textbf{Ajax}} et les représentations graphiques sur un site web. Privilégier le langage \textit{\textbf{JavaScript}} au détriment de \textit{\textbf{Python}} pour le traitement des données aurait limité les échanges avec le serveur, ce qui augmente aussi les performances. Néanmoins, et dans le cadre pédagogique de ce projet, les délais imposés ne laissaient pas suffisamment de marge pour une maîtrise de ces vastes outils dont la prise en main n'est pas rapide.
	\paragraph{} Ce travail a été réalisé par un groupe de 3 personnes. Le sujet a été très bien appréhendé par les membre du groupe et l'entente a été excellente. Grâce aux informations préalablement établies dans le cahier des charges et dans les spécifications, le travail au sein du groupe a pu être mené de façon assez indépendante, sans générer de conflits ni de problèmes d'organisation.
	\paragraph{} Cette épreuve constitue donc une expérience enrichissante pour nous, non seulement sur le plan compétence, mais aussi sur le plan de travail collectif, qui nous inspirera certainement pour de futurs projets.
		
\end{document}
