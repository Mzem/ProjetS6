\input{../tex/preambule}

\title{\vspace{\fill}\textbf{\Huge Compte Rendu}}
\author{
	Sonny Klotz - Jean-Didier Pailleux - Malek Zemni
	\vspace{2em}\\
	\textit{Interface de chargement, de contrôle}\\\textit{et d’analyse statistique des données}\\\textit{pour la constitution d’un graphe de flux}
	\vspace{2em}
}

\begin{document}
\pagenumbering{gobble}\clearpage
\maketitle\vspace{9em}
\begin{center}\includegraphics[scale=0.7]{../Cahier/logo.png}\end{center}
\begin{flushright}Module \textit{Projet}\end{flushright}
\newpage
\tableofcontents
\newpage\clearpage\pagenumbering{arabic}

	\section*{Introduction}
		Motivations du projet : un mix de
			- Description Sujet (notre premier oral)
			- Intro et partie 1 cahier charges
		
		Plan :
			- archi appli (cahier charges) + données transitent + fonctionnalités
			- langages de programmation : découlant des contraintes de l'appli
			- bilan technique
			- organisation interne
			
		Documents fournis en plus du code :
			- Manuel
			- Documentation
			
		-> 1 page (Sonny)
	
	\section{Architecture de l'application}
		\subsection{Organigramme et données échangées}
			Phrase de présentation -> diagramme -> légende (non modifiée, on la met en accord avec le dernier produit quand on aura finit)
		\subsection{Format du fichier CSV}
			une demi-page
		\subsection{Fonctionnalités des modules}
			Les fonctionnalités de notre produit final (pas ceux du cahier des charges)
		
		-> entre 3 pages (pas plus) pour la partie (Sonny)
		
	\section{Contraintes et Langages de programmation}
		contraintes :
			-voir cahier des charges : applet, add, ...
		choix langages et frameworks:
			-python : justification + flask (voir cahier + ton expérience) -> intègre en plus un module de tests
			-html / css : easy on fait du web
			-javascript - jquery et AJAX: UX design tu peux parler vite fait, js donne intéractivité coté client, ajax pour communiquer infos côté serveur entre autres et jquery qui permet de faire tout ca
			-Sphinx, génération documentation automatique (la doc flask est générée grace a sphinx par ex), si tu sais pas on le fera plus tard, mais faut pas l'oublier
			
		-> 2 pages (J-D)
		
	\section{Partie technique du projet}
		Quoi marche (comment ca marche - à l'aide de la partie avant)
		QUoi marche pas (pb et et resolution si trouvee)
		
		1 page, plus tard
		
	\section{Points délicats}
		
		Justification des points qui divergent par rapport aux specs et organigramme de départ
		
		1 page, plus tard à moins que t'ai déjà des idées
				
	\section{Organisation interne du groupe}
		
		Insérer la répartition des tâches (cahier des charges)
		
		-> (J-D)
		
		1 page
		
		On expliquera plus tard comment ca s'est déroulé concrètement par rapport à ce quiétait prévu.
		
	\section{Coût}
		
		Insérer le tableau des couts, ajouter une colonne (vide) "coût" le final
		
		-> J-D
		1 demi - page
		
	\section*{Conclusion}
		
		1 demi-page, plus tard
		
\end{document}
