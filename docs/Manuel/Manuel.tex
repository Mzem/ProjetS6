\input{../tex/preambule}

\title{\vspace{\fill}\textbf{\Huge Manuel utilisateur}}

\begin{document}
\pagenumbering{gobble}\clearpage
\maketitle\vspace{9em}
\begin{center}\includegraphics[scale=0.7]{../Cahier/logo.png}\end{center}
\begin{flushright}Module \textit{Projet}\end{flushright}
\newpage
\tableofcontents
\newpage\clearpage\pagenumbering{arabic}

\section{Introduction}
	L'applet est un outil d'analyses descriptives de données. Elle se manipule à l'aide d'un navigateur web et de fichiers \lstinline!.csv! compatibles.\\
	La navaigation entre les différentes fonctionnalités de l'applet ainsi que le format des fichiers compatibles pour les analyses sont précisées dans ce document.
	
\section{Guide Pratique}
	\subsection{Matériel nécessaire}
		Navigateur web compatible - (préciser lesquels dans quelle version plus tard)
	\subsection{Accès à l'application}
	\subsection{Étape 1-Sélection d'un fichier}
		- Parler aussi vite fait du format csv (Timestamp - Parent Enfant - Mesures)
	\subsection{Étape 2-Vérification et validation}
	\subsection{Étape 3-Choix et filtrage de la colonne}
	\subsection{Étape 4-Résultats}
			(Insérez un screenshot)\\
		La fenêtre que vous pouvez apercevoir ci-dessus représente la dernière étape du traitement des données, l'étape de consultation des résultats d'analyse descriptive.\\
		Ces résultats sont constitués de deux catégories d'éléments :
		\begin{itemize}
			\item Des valeurs statistiques à consulter directement.
			\item Des représentations graphiques :\\
				Il suffit de cliquer sur une miniature pour accéder à une vue précise et interactive du graphe sélectionné.\\
			(Insérez un screenshot - avant / après le clic)			
		\end{itemize}
		Finalement, une fonctionnalité de retour est disponible pour revenir à l'étape 3 et effectuer une autre analyse sans relancer l'applet du début.\\
			(Insérez un screenshot)
\end{document}
