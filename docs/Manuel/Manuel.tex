\input{../tex/preambule}

\title{\vspace{\fill}\textbf{\Huge Manuel utilisateur}}

\begin{document}
\pagenumbering{gobble}\clearpage
\maketitle\vspace{9em}
\begin{center}\includegraphics[scale=0.7]{../Cahier/logo.png}\end{center}
\begin{flushright}Module \textit{Projet}\end{flushright}
\newpage
\tableofcontents
\newpage\clearpage\pagenumbering{arabic}

\section{Introduction}
	L'applet est un outil d'analyses descriptives de données. Elle se manipule à l'aide d'un navigateur web et de fichiers \lstinline!.csv! compatibles.\\
	La navaigation entre les différentes fonctionnalités de l'applet ainsi que le format des fichiers compatibles pour les analyses sont précisées dans ce document.
	
\section{Guide Pratique}
	\subsection{Matériel nécessaire}
		Navigateur web compatible
	\subsection{Accès à l'application}
	\subsection{Étape 1-Sélection d'un fichier}
		- Parler aussi vite fait du format csv (Timestamp - Parent Enfant - Mesures)
	\subsection{Étape 2-Vérification et validation}
	\subsection{Étape 3-Choix et filtrage de la colonne}
	\subsection{Étape 4-Résultats}
		
	
\end{document}
